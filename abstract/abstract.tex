%!TEX root = ../main.tex

\newgeometry{top=1.25in, bottom=1.0in, left=1.5in, right=1.5in}
\begin{abstract}
  In modern engineering, the accurate and efficient numerical simulation of dynamic systems is crucial, providing valuable insights across various fields such as automotive, aerospace, robotics, and electrical engineering. These simulations help to understand system behaviors, to optimize performance, and to guide design decisions. Nonetheless, systems described by \acp{ODE} and \acp{DAE} are central to such simulations. While \acp{ODE} can be easily solved, they often fall short of modeling systems with constraints or algebraic relationships. \acp{DAE}, however, offer a more comprehensive framework, making them suitable for a wider range of dynamic systems. However, the inherent complexity of \acp{DAE} poses significant challenges for numerical integration and solution.

  In vehicle dynamics, the simulation of systems described by \acp{DAE} is particularly relevant. The advances on \ac{ADAS}, autonomous vehicles, and high-performance cars rely heavily on robust simulations that accurately reflect the interactions between mechanical components, control systems, and environmental factors. Achieving accuracy and speed in these simulations is critical for \ac{RT} applications, where rapid decision-making and control are essential. The challenges faced in vehicle dynamics simulations, such as equations' stiffness and complexity, are representative of broader issues in dynamic system simulations.

  This thesis addresses these challenges by integrating symbolic computation with numerical methods to solve \acp{DAE} efficiently and accurately. Specifically, the index reduction approach transforms high-index \acp{DAE} into low-index formulations more suitable for numerical integration, enhancing the speed and stability of solvers. Symbolic computation, which handles mathematical expressions in their exact form, aids this process by simplifying the involved expressions, detecting redundancies and symbolic cancellations, and thereby ensuring equations' consistency while keeping complexity at the minimum. Hence, combining symbolic and numerical methods leverages the strengths of both techniques, aiming at improved performance and reliability. This hybrid framework is designed to handle the specific requirements of vehicle dynamics and other applications in engineering.

  The primary objectives of this work focus on advancing computational methods for the simulation and the analysis of complex dynamic systems, particularly in vehicle dynamics. One key goal is to enhance numerical methods through the integration of symbolic computation, aiming to develop a robust framework that combines these approaches to reduce computational overhead and improve performances. Specifically, the research seeks to create new algorithms for \acp{DAE}' index reduction by developing a novel symbolic algorithm that transforms high-index \acp{DAE} into formulations more suitable for standard numerical methods, while also addressing expression swell phenomenon. This thesis further seeks to validate such a methodology through practical applications. For this reason, it focuses on applying the proposed framework and algorithms to real-world simulation problems to assess its performance, accuracy, and efficiency. Additionally, the research also aims to develop sub-models and techniques for \ac{HRT} vehicle dynamic simulation. This includes the design of dedicated algorithms and models for simulating tire-road interactions and vehicle structures, which are then integrated into the simulation to improve both speed and fidelity.

  This thesis encompasses several advancements in symbolic computation and dynamic system simulation. Notably, it develops advanced techniques for symbolic matrix factorization with expression swell mitigation, enhancing computational efficiency and scalability. A novel algorithm for \ac{DAE} systems index reduction is introduced, transforming high-index \acp{DAE} into formulations more suitable for numerical integration. Furthermore, specialized models for \ac{HRT} tire-road interaction are developed, focusing on the dynamics and physical modeling of tires to enable accurate and \ac{RT} simulations. The research also extends the \ac{DSM} to mixed symbolic-numerical analysis of structures, allowing for efficient assembly and solution of structural systems. Nonetheless, the results are several open-source software libraries, which are made available to the research community with comprehensive documentation and examples.

  In summary, this work bridges the gap between symbolic computation and numerical methods for the simulation of dynamic systems described by \acp{DAE}. Thanks to mixed symbolic-numeric frameworks, innovative algorithms, and practical tools, it contributes to the advancement of simulation techniques, setting the stage for further investigations and applications in engineering.
\end{abstract}
\restoregeometry
\acresetall
