%!TEX root = ../main.tex

\newgeometry{top=1.25in, bottom=1.0in, left=1.5in, right=1.5in}
\begin{abstract}
  The accurate and efficient numerical simulation of dynamic systems is crucial in modern engineering, providing valuable insights across various fields such as automotive, aerospace, robotics, and electrical circuits. These simulations help understand system behaviors, optimize performance, and guide design decisions. With the rise of \ac{AI} and machine learning, simulations have become even more critical for training and validating models and control algorithms. Systems described by \acp{ODE} and \acp{DAE} are central to these simulations. While \acp{ODE} are straightforward to solve, they often fall short in modeling systems with constraints or algebraic relationships. \acp{DAE} offer a more comprehensive framework by combining differential equations with algebraic constraints, making them suitable for a wide range of dynamic systems. However, the inherent complexity of \acp{DAE} poses significant challenges for numerical integration and solution.

  In vehicle dynamics, the simulation of systems described by \acp{DAE} is particularly important. The advances on \ac{ADAS}, autonomous vehicles, and high-performance cars rely heavily on robust simulations that accurately reflect the interactions between mechanical components, control systems, and environmental factors. Achieving accuracy and speed in these simulations is critical for \ac{RT} applications, where rapid decision-making and control are essential. The challenges faced in vehicle dynamics simulations, such as stiffness and complexity, are representative of broader issues in dynamic system simulations.

  This research addresses these challenges by integrating symbolic computation with numerical methods to solve \acp{DAE} efficiently and accurately. Symbolic computation handles mathematical expressions in their exact form, which is beneficial for managing structural complexities like index reduction. This approach can transform high-index \acp{DAE} into forms more suitable for numerical integration, enhancing the speed and stability of solvers. Hence, combining symbolic and numerical methods leverages both strengths, aiming for improved performance and reliability. This hybrid framework is designed to handle the specific requirements of vehicle dynamics and other engineering applications.

  The primary objectives of this research focus on advancing computational methods for simulating and analyzing complex dynamic systems, particularly in vehicle dynamics. One key goal is to enhance numerical methods by integrating symbolic computation, aiming to develop a robust framework that combines these approaches to solve stiff high-index \acp{DAE}. This involves preprocessing and simplifying equations to reduce computational overhead and improve solver stability and performance. Additionally, the research seeks to create new algorithms for DAE index reduction by developing novel symbolic algorithms that transform high-index \acp{DAE} into forms more suitable for standard numerical methods, addressing expression swell—a common issue in symbolic computation—and expanding the applicability of numerical solvers. The validation of these methods through practical applications is also a crucial objective, with a focus on applying the framework and algorithms to real-world simulation problems, such as vehicle dynamics, to assess their performance, accuracy, and efficiency. Furthermore, the research aims to develop sub-models and techniques specifically for \acp{HRT} vehicle simulations, including dedicated algorithms and models for simulating tire-road interactions and vehicle structures, which are integrated into the overall simulation framework to enhance both speed and accuracy.

  Key contributions of this research encompass several advancements in symbolic computation and dynamic system simulation. Notably, the research has developed advanced techniques for symbolic matrix factorization with expression swell mitigation, enhancing computational efficiency and scalability. Additionally, novel algorithms for \acp{DAE} index reduction are introduced, transforming high-index \acp{DAE} into forms more suitable for numerical integration, and implemented in a dedicated software library. These techniques are implemented in the \Maple{} \ac{CAS}. Specialized models for \acp{HRT} tire-road interaction are developed, focusing on the dynamics and physical modeling of tires to enable accurate and real-time simulations. The research also extends the \acp{DSM} for the symbolic-numerical analysis of structures, allowing for efficient assembly and solution of large-scale structural systems. Moreover, the project has resulted in several user-friendly software libraries designed to promote wider adoption and further development of advanced computational methods. These libraries, made available to the research community with comprehensive documentation and examples, aim to facilitate their use in various applications, encouraging collaboration and innovation in the dynamic system simulation field.

  In summary, this research bridges the gap between symbolic computation and numerical methods for solving \acp{DAE} in vehicle simulations. Thanks to mixed symbolic-numeric frameworks, innovative algorithms, and practical tools, it contributes to the advancement of simulation techniques for complex dynamic systems, setting the stage for further exploration and application in engineering.
\end{abstract}
\restoregeometry
\acresetall