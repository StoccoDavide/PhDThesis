%!TEX root = ../main.tex

\newgeometry{top=1.25in, bottom=1.0in, left=1.5in, right=1.5in}
\begin{abstract}

  \acp{DAE} find extensive application in modeling complex systems including electrical circuits, mechanical systems, chemical reactions, as well as in the solution of trajectory-prescribed path control problems. The differential index of a \ac{DAE} system, indicating the minimum number of differentiations needed to transform it into an equivalent system of \acp{ODE}, serves as an indicator of the system's solution difficulty. Index reduction is a crucial preliminary step in the numerical integration of \acp{DAE}, which reduces the system's complexity significantly and allows the usage of standard \acp{ODE} solvers. Notably, index reduction is critical for addressing high-index \ac{DAE} systems. Conversely, if index reduction is not performed, high-order numerical solvers are required to just achieve a low-precision solution. Although closed-form index reduction methods exist for specific \ac{DAE} classes like multi-body systems, reducing the index of generic first-order \ac{DAE} systems can pose challenges, particularly for large systems with many equations and variables.

  This research introduces a novel algorithm designed to reduce the index of generic first-order \acp{DAE}, which are linear in the states' derivatives, employing \acp{CAS} for symbolic equation manipulation. The algorithm uses a repeated application of symbolic matrix factorization to automatically separate differential and algebraic equations, followed by the symbolic differentiation of the separated algebraic equations to decrease by one the \acp{DAE} index. These steps are repeated until deemed necessary, typically until an index-0 \acp{DAE} (\acp{ODE}) or index-1 \acp{DAE} is obtained. Special attention is given to the mitigation of the so-called expression swell, a common issue in symbolic computation that can lead to excessive memory usage and slow computation times. An effective strategy -- based on the introduction of hierarchical index-1 variables -- is proposed to address this issue that affects the symbolic matrix factorization process. This strategy, aimed at easing the computational burden and time, leads to advancements in both symbolic matrix factorization and \acp{DAE} index reduction techniques.

  The proposed methodology is implemented in the open-source \Indigo{} library, employed for symbolic manipulation of \acp{DAE} in the \Maple{} environment, as well as for generating optimized code for numerical integration of the reduced system in \Matlab{}. Dependencies on external libraries are limited to the \LEM{} and \LAST{} \Maple{} open-source packages, which perform the symbolic matrix factorization with large expression management. Validation of the algorithm encompasses various benchmark problems, spanning mechanical systems, electrical circuits, trajectory-prescribed path control problems, and artificial \acp{DAE} with arbitrarily high indexes. The results demonstrate its effectiveness in reducing the index of such systems, highlighting its robustness across several applications. Furthermore, the generated code for numerical integration exhibits stability and efficiency, showcasing the algorithm's practical utility in real-world scenarios. This capability is particularly significant in the context of complex systems, where reducing the index of generic \acp{DAE} is crucial for accurate simulations. Overall, the proposed method provides users with a reliable means to address the index-reduction of high-index \acp{DAE}, offering a valuable tool for effectively tackling the numerical solution of complex dynamical systems.
\end{abstract}
\restoregeometry