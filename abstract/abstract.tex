%!TEX root = ../main.tex

%\newgeometry{top=2.0cm, bottom=2.0cm, left=2.5cm, right=2.5cm}
\begin{abstract}
  In modern engineering, the accurate and efficient numerical simulation of dynamic systems is crucial, providing valuable insights across various fields such as automotive, aerospace, robotics, and electrical engineering. These simulations help to understand system behaviors, to optimize performance, and to guide design decisions. Nonetheless, systems described by \acp{ODE} and \acp{DAE} are central to such simulations. While \acp{ODE} can be easily solved, they often fall short of modeling systems with constraints or algebraic relationships. \acp{DAE}, however, offer a more comprehensive framework, making them suitable for a wider range of dynamic systems. However, the inherent complexity of \acp{DAE} poses significant challenges for numerical integration and solution.

  In vehicle dynamics, the simulation of systems described by \acp{DAE} is particularly relevant. The advances in autonomous and high-performance cars rely heavily on robust simulations that accurately reflect the interactions between mechanical components, control systems, and environmental factors. Achieving accuracy and speed in these simulations is critical for \ac{RT} applications, where rapid decision-making and control are essential. The challenges faced in vehicle dynamics simulations, such as equations' stiffness and complexity, are representative of broader issues in dynamic system simulations.

  This thesis addresses these challenges by integrating symbolic computation with numerical methods to solve \acp{DAE} efficiently and accurately. Specifically, the index reduction approach transforms high-index \acp{DAE} into low-index formulations more suitable for numerical integration, enhancing the speed and stability of solvers. Symbolic computation, which handles mathematical expressions in their exact form, aids this process by simplifying the involved expressions, detecting redundancies and symbolic cancellations, and thereby ensuring equations' consistency while keeping complexity at the minimum. Hence, combining symbolic and numerical methods leverages the strengths of both techniques, aiming at improved performance and reliability. Such a hybrid framework is designed to handle the specific requirements of vehicle dynamics and other applications in engineering.

  The thesis encompasses several advancements in dynamic system simulation by integrating symbolic computation with numerical methods to reduce computational overhead and improve performance. The research focuses on developing new algorithms for \acp{DAE} index reduction, transforming high-index \acp{DAE} into more suitable for standard numerical integration methods. Specifically, such an index reduction process is based on symbolic matrix factorization with simultaneous expression swell mitigation. This novel methodology is validated through practical applications, applying the proposed technique to real-world simulation problems to assess its performance, accuracy, and efficiency. Additionally, the research aims to enhance \ac{HRT} vehicle dynamic simulation by designing dedicated algorithms and models for simulating tire-road interactions and vehicle structures' deformation, improving both speed and fidelity. Altogether, this thesis introduces several open-source software libraries made available to the research community with comprehensive documentation and examples.

  In summary, this work bridges the gap between symbolic computation and numerical methods for the simulation of dynamic systems described by \acp{DAE}. Thanks to mixed symbolic-numeric frameworks, innovative algorithms, and practical tools, it contributes to the advancement of simulation techniques, setting the stage for further investigations and applications in engineering.
\end{abstract}
%\restoregeometry
\acresetall
