%!TEX root = ../main.tex

\newgeometry{top=1.25in, bottom=1.0in, left=1.5in, right=1.5in}
\begin{abstract}
  \acp{DAE} find extensive application in modeling complex dynamical systems, including electrical circuits, mechanical systems, and chemical reactions. However, the solution of \acp{DAE} is not as straightforward as that of \acp{ODE}, as they involve both differential and algebraic equations, which complicates their numerical integration. The differential index of a \ac{DAE} system, indicating the minimum number of differentiations needed to transform it into an equivalent system of \acp{ODE} with invariants, serves as an indicator of the system's solution difficulty. Therefore, reducing the index is a crucial preliminary step in the numerical integration of high-index  \acp{DAE}, which lowers the system's complexity and allows the usage of standard \acp{ODE} solvers. Nonetheless, if index reduction is not performed, high-order numerical solvers are required even to achieve a low-precision solution. Although closed-form index reduction methods exist for specific \ac{DAE} classes, like multi-body systems or electrical circuits, reducing the index of generic first-order \acp{DAE} poses challenges, especially for large systems with many equations and variables.

  This research introduces a novel algorithm designed to reduce the index of generic first-order \acp{DAE}, which are linear in the states' derivatives, employing \acp{CAS} for symbolic equation manipulation. The algorithm uses symbolic matrix factorization to automatically separate differential and algebraic equations, followed by the symbolic differentiation of the separated algebraic equations to decrease by one the \acp{DAE} index. These two steps repeat until deemed necessary, and typically an index-0 \acp{DAE} (\acp{ODE} with invariants) or index-1 \acp{DAE} is obtained from such a process. Special attention is given to the mitigation of the so-called expression swell phenomenon, a common issue in symbolic computation that leads to excessive memory usage and slow computation times. An effective strategy based on hierarchical index-1 variables addresses this issue, which affects the symbolic matrix factorization process. This strategy, aimed at easing the computational burden and time, is a crucial feature of the algorithm and represents a significant advancement in both symbolic matrix factorization and \acp{DAE} index reduction techniques.

  The proposed methodology is implemented in the open-source \Indigo{} library, employed for symbolic manipulation of \acp{DAE} in the \Maple{} environment, as well as for generating optimized code for numerical integration of the reduced system in \Matlab{}. Dependencies on external libraries are limited to the \LEM{} and \LAST{} \Maple{} open-source packages, which perform the symbolic matrix factorization with large expression management. Validation of the algorithm encompasses various benchmark problems, spanning mechanical systems, electrical circuits, trajectory-prescribed path control problems, and artificial \acp{DAE} with arbitrarily high index. The results demonstrate its effectiveness in symbolically reducing the index of such systems, highlighting its robustness across several applications. Furthermore, the generated code for numerical integration exhibits stability and efficiency, showcasing the algorithm's practical utility in real-world scenarios. This capability is particularly significant in the context of complex systems, where reducing the index of generic \acp{DAE} is crucial for accurate and fast simulations. Overall, the proposed method provides users with a reliable means to address the index-reduction of high-index \acp{DAE}, offering a valuable tool for effectively tackling the numerical solution of complex dynamical systems.
\end{abstract}
\restoregeometry