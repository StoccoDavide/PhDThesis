%!TEX root = ../main.tex

\newgeometry{top=1.25in, bottom=1.0in, left=1.5in, right=1.5in}
\begin{abstract}
  The accurate and efficient numerical simulation of dynamic systems is crucial in modern engineering, providing valuable insights across various fields such as automotive, aerospace, robotics, and electrical engineering. These simulations help to understand system behaviors, optimize performance, and guide design decisions. Nonetheless, systems described by \acp{ODE} and \acp{DAE} are central to such simulations. While \acp{ODE} are straightforward to solve, they often fall short in modeling systems with constraints or algebraic relationships. \acp{DAE} offer a more comprehensive framework by combining differential equations with algebraic constraints, making them suitable for a wide range of dynamic systems. However, the inherent complexity of \acp{DAE} poses significant challenges for numerical integration and solution.

  In vehicle dynamics, the simulation of systems described by \acp{DAE} is particularly relevant. The advances on \ac{ADAS}, autonomous vehicles, and high-performance cars rely heavily on robust simulations that accurately reflect the interactions between mechanical components, control systems, and environmental factors. Achieving accuracy and speed in these simulations is critical for \ac{RT} applications, where rapid decision-making and control are essential. The challenges faced in vehicle dynamics simulations, such as stiffness and complexity, are representative of broader issues in dynamic system simulations.

  This research addresses these challenges by integrating symbolic computation with numerical methods to solve \acp{DAE} efficiently and accurately. Symbolic computation handles mathematical expressions in their exact form, which is beneficial for lowering structural complexities through index reduction. This approach can transform high-index \acp{DAE} into forms more suitable for numerical integration, enhancing the speed and stability of solvers. Hence, combining symbolic and numerical methods leverages both strengths, aiming for improved performance and reliability. This hybrid framework is designed to handle the specific requirements of vehicle dynamics and other engineering applications.

  The primary objectives of this research focus on advancing computational methods for simulating and analyzing complex dynamic systems, particularly in vehicle dynamics. One key goal is to enhance numerical methods by integrating symbolic computation, aiming to develop a robust framework that combines these approaches to reduce computational overhead and improve performances. Specifically, the research seeks to create new algorithms for \acp{DAE} index reduction by developing novel symbolic algorithms that transform high-index \acp{DAE} into forms more suitable for standard numerical methods, addressing expression swell -- a common issue in symbolic computation -- and expanding the applicability of numerical solvers. The validation of these methods through practical applications is also a crucial objective, focusing on applying the framework and algorithms to real-world simulation problems, such as vehicle dynamics, to assess their performance, accuracy, and efficiency. Additionally, the research aims to develop sub-models and techniques specifically for \ac{HRT} vehicle simulations, including dedicated algorithms and models for simulating tire-road interactions and vehicle structures, which are integrated into the overall simulation framework to enhance both speed and accuracy.

  Key contributions of this research encompass several advancements in symbolic computation and dynamic system simulation. Notably, the research develops advanced techniques for symbolic matrix factorization with expression swell mitigation, enhancing computational efficiency and scalability. Novel algorithms for \ac{DAE} systems index reduction are introduced, transforming high-index \acp{DAE} into forms more suitable for numerical integration. Furthermore, specialized models for \ac{HRT} tire-road interaction are developed, focusing on the dynamics and physical modeling of tires to enable accurate and \ac{RT} simulations. The research also extends the \ac{DSM} to mixed symbolic-numerical analysis of structures, allowing for efficient assembly and solution of large-scale structural systems. Nonetheless, the project results in several open-source software libraries, made available to the research community with comprehensive documentation and examples.

  In summary, this research bridges the gap between symbolic computation and numerical methods for the simulation of dynamic systems described by \acp{DAE}. Thanks to mixed symbolic-numeric frameworks, innovative algorithms, and practical tools, it contributes to the advancement of simulation techniques, setting the stage for further investigations and application in engineering.
\end{abstract}
\restoregeometry
\acresetall
