%!TEX root = ../main.tex

\chapter{A Small 3D Geometry Library}
\label{app1:acme}

In the past few decades, the simulation of both manned and unmanned vehicles has gained increasing significance. The demand for highly efficient \ac{RT} simulators underscores the necessity for algorithms that are not only efficient but also accurate in modeling vehicle movements within a virtual environment. Typically, the virtual world comprises numerous basic geometric entities capable of colliding and adjusting accordingly. The ability to quickly solve basic geometric problems is one of the most important roles in this kind of simulation. The \Acme{} library, previously introduced in \citet{stocco2021acme}, is built to efficiently perform simple operations on a large number of basic geometric entities. Specifically, the library is tailored to address specific \ac{HRT} tire-ground contact geometry analysis. This chapter describes the implementation details of the \Acme{} library, exploring its data types, features, and ease of use. The software is implemented in \cpp{} and is freely available online under the \ac{BSD} 2-Clause license. Online documentation includes descriptions of the \cpp{} and \Matlab{} \Mex{} \acp{API}, along with usage examples.

% % % % % % % % % % % % % % % % % % % % % % % % % % % % % % % % % % % % % % % %

\section{Computational Geometry in Real-Time Simulations}
\label{app1:sec:acme_motivation}

Over the recent decades, there has been a notable shift in the automotive manufacturing sector towards prioritizing simulation. In particular, the advent of high-performance \acp{CPU} and \acp{GPU} has intensified the focus on \ac{RT} simulators. These sophisticated simulators, characterized by high fidelity and full integration, demand substantial computational capabilities. Moreover, specialized codes are essential to meet the dual requirements of \ac{HRT} responsiveness and high accuracy within very limited time intervals. Indeed, the time step in driving simulators is typically set at \SI{1}{\milli\second}, which is a trade-off to capture most of the typical frequencies in vehicle subsystems.

An important aspect of simulation lies in the vehicle-environment interaction. In driving simulators the virtual environment on which the vehicle moves is made from a multitude of basic geometric entities that can intersect and evolve. Consequently, the efficient resolution of simple geometric problems assumes a crucial role in achieving high accuracy and, by extension, a realistic simulation. Specifically, when working with numerous geometric objects, it is essential to partition the 3D space with an appropriate data structure. This structured partitioning facilitates efficient access to spatial objects. The absence of spatial partitioning would necessitate scanning the entire database during any search, resulting in a significant increase in processing time.

There is a multitude of geometric libraries already implemented and capable of solving complicated geometric problems, \eg{}, mesh-mesh intersection, re-meshing, Delaunay triangulation, and so on. The sheer size of such libraries and their high complexity make them unsuitable for application in the simulation environment introduced earlier. The need to easily maintain and correct inefficiencies has led to the development of a new geometry library. In this chapter, we introduce a \cpp{} library named \Acme{}, designed to efficiently address the resolution of basic 3D geometric problems at high speed. The first version of \Acme{} was tailor-made to perform \ac{HRT} tire-ground contact geometry analysis, where geometrical objects and tire-ground intersection objects were initially coexistent in the same code. The desire to bring the library to the next level made it necessary to formalize and create a more effective framework. Consequently, all the geometrical algorithms are then collected in an independent library. But why create a new library even if there are plenty of alternatives available out there? The dynamic nature of the simulation field, with the continual introduction of new features, underscores the importance of maintaining a simple yet robust minimum core. This approach enables quick response to changes. Most of the available geometry libraries are either excessively large or overly complex for this specific purpose~\cite{cgal2023cgal, libigl2018libigl}. Furthermore, we aim to reduce the dependencies by relying solely on the \cpp{} \Eigen{} template library, which is well-recognized for its efficiency with small vectors and matrices.

% % % % % % % % % % % % % % % % % % % % % % % % % % % % % % % % % % % % % % % %

\section{A New Geometry Library}

As previously mentioned, the software is implemented in \cpp{}, a widely used and high-performance object-oriented general-purpose programming language. Since its invention by Bjarne Stroustrup in 1985, \cpp{} has undergone significant extensions and modifications. Therefore, we chose to develop our code based on the \cpp{}11 standard~\cite{stroustrup2013cpp}. The adoption of the 2011 standard introduced notable improvements in the coding style, exemplified by the introduction of the new smart pointer classes, extensively utilized in the \Acme{} library. The source code of the software is freely available online~\cite{acme} and is released under the \ac{BSD} 2-Clause license. The online documentation includes descriptions of the \cpp{} and \Matlab{} \Mex{} \acp{API}, along with usage examples. Rigorous testing has been conducted on \MacOS{}, \Linux{}, and \Windows{} \acp{OS} to ensure the software's compatibility across diverse platforms.

\subsection{Design Choices}

This software is neither intended as a black box nor as a \ac{GUI} based application for end-users. Instead, it is designed as an easy-to-use set of \cpp{} classes that provides a basic and reliable foundation, which can be extended by the developers according to their specific needs. The design of the software is grounded in the following principles.

\paragraph{Driven by Actual Needs}
The implementation focuses on a stable minimum core library, including only features that are currently in use. This deliberate choice allows for progressive testing of the software and a less concerned third-party extension process.

\paragraph{Build on the State-of-the-Art \Eigen{} Library}
For a flexible and extensible framework, \Acme{} is built on the \Eigen{} template library for linear algebra~\cite{eigen2010eigen}. Recognized for its efficiency with small vectors and matrices, \Eigen{} is an apt choice in the field of computational geometry where matrices and vectors are typically of limited size. Additionally, \Eigen{} can leverage \ac{LAPACK}/\ac{BLAS}~\cite{anderson1999lapack} for peak performance when dealing with larger matrices and vectors. Relying on this well-tested and high-performing template library allows \Acme{} to achieve high-performance levels while maintaining an elegant and expressive \ac{API}.

\paragraph{Avoiding Memory Leaks}
Managing dynamically allocated memory is one of the most critical aspects of a low-level programming language like \cpp{}. Often, the most insidious errors are due to flaws in memory allocation and release policies, resulting in excessive use of resources (\emph{memory leak}), or irreversible error conditions that undermine program stability (\emph{access violation}). The usage of \cpp{}11 smart pointers in \Acme{} significantly reduces the likelihood of these errors. Smart pointers, as part of the standard library utility classes, act as wrappers for raw pointers, offering transparent memory release policies suitable for various use cases. Notably, \SharedPointer{} objects retain shared ownership, allowing multiple objects to own the same instance. The object is only destroyed and its memory deallocated when either the last \SharedPointer{} owning it is destroyed or reassigned to another \SharedPointer{}. Additionally, the object can be destroyed using the \code{delete} expression or a custom \code{delete} expression.

\paragraph{Polymorphic Behavior}
\Acme{} capitalizes on \cpp{} polymorphism as a fundamental design pattern. This polymorphic behavior greatly simplifies the management of heterogeneous objects that share a common interface of geometric entities. Notably, the same \cpp{} polymorphic behavior is also present in the \Matlab{} \Mex{} wrapper.

\paragraph{High-quality Documentation}
Comprehensive documentation is available on the provided website, encompassing both the \cpp{} and \Matlab{} \Mex{} \acp{API}, along with examples. The documentation is generated using a combination of \Doxygen{} and \Sphinx{}. \Doxygen{} processes annotated \cpp{} sources to create documentation, while \Sphinx{} enhances the graphical quality of the generated \html{} code, providing a more visually appealing and graphically rich design.

\subsection{Data Types}

\Acme{} supports a limited number of geometrical entities, carefully chosen to maintain the library's essential nature for efficiency and easy maintenance. The chosen classes specifically describe and manipulate virtual ground surfaces and tires. While the library is intentionally kept minimal, it is extensible according to the needs of end-users. The geometric entities are systematically organized into classes, each being within the \Acme{} namespace and publicly inheriting from the virtual superclass \Entity{}. The derived classes, representing the homonyms geometric entities are \Point{}, \Line{}, \Ray{}, \Plane{}, \Segment{}, \Triangle{}, \Disk{}, and \Ball{}, are integral components of the library. In Figure~\ref{app1:fig:acme_entities}, a representation of all \Acme{} basic \Entity{} objects is shown. A concise mathematical description of each data type in the software follows.

\begin{figure}[htb]
  \centering
  \def\svgwidth{8.0cm}
  \input{./figures/appendix_1/acme_entities_1.pdf_tex} \\[1.0em]
  \def\svgwidth{8.0cm}
  \input{./figures/appendix_1/acme_entities_2.pdf_tex}
  \caption{Representation of all \Acme{} basic \Entity{} objects.}
  \label{app1:fig:acme_entities}
\end{figure}

\paragraph{\Point{}}
In the 3D Euclidean space, a point represents an exact location. A point $\pt{p} \in \mathbb{R}^3$ is represented by an ordered triplet of coordinates
%
\begin{equation*}
  \pt{p} = \left[x, y, z\right]^{\top} \, \text{.}
\end{equation*}
%
The \Point{} class is built through public inheritance from the virtual class \Entity{} and the \Eigen{}::\MatrixBase{} template class. It is worth noting that in many \cpp{} libraries, vectors and points are often described by the same class. However, in \Acme{}, we have provided a clear way to distinguish them. This distinction is evident in the inheritance structure, where the \Point{} class inherits publicly from the \Entity{} class. On the other hand, the inheritance of the \MatrixBase{} template class makes it possible to easily build mathematical vectors out of point entities and vice versa. In our software, both vectors and points are represented by column sets of elements.

\paragraph{\Line{}}
A line $\sett{\ell}$ is defined by an origin point $\pt{o}$ and a unit direction vector $\et{d}$, such that the line corresponds to the set
%
\begin{equation*}
  \sett{\ell}(\pt{o}, \et{d}) = \left\{ \pt{o} + \et{d} t ~ \big| ~ t \in \mathbb{R} \right\} \, \text{.}
\end{equation*}

\paragraph{\Ray{}}
A ray $\sett{\varrho}$ is defined by an origin point $\pt{o}$ and a unit direction vector $\et{d}$, such that the ray corresponds to the set
%
\begin{equation*}
  \sett{\varrho}(\pt{o}, \et{d}) = \left\{ \pt{o} + \et{d} t ~ \big| ~ t \in \mathbb{R}_{\ge0} \right\} \, \text{.}
\end{equation*}

\paragraph{\Plane{}}
A plane ${\pi}$ is defined by a generic point on the plane $\pt{p}$ and a unit normal vector $\et{n}$, such that the plane corresponds to the set
%
\begin{equation*}
  {\pi}(\pt{p}, \et{n}) = \left\{ \et{n} \cdot (\pt{p} - \left[x, y, z\right]^{\top}) = 0 ~ \big| ~ \left[x, y,  z\right]^{\top} \in \mathbb{R}^3 \right\} \, \text{.}
\end{equation*}

\paragraph{\Segment{}}
A segment $\sett{\sigma}$ is defined by two points $\pt{p}_1$ and $\pt{p}_2$, such that the segment corresponds to the set
%
\begin{equation*}
  \sett{\sigma}(\pt{p}_1, \pt{p}_2) = \left\{ \pt{p}_1 + \left(\pt{p}_2-\pt{p}_1\right)t ~ \big| ~ t \in \mathbb{R}, \, 0 \le t \le 1 \right\} \, \text{.}
\end{equation*}

\paragraph{\Triangle{}}
A triangle $\sett{\tau}$ is defined by three points $\pt{p}_1$, $\pt{p}_2$ and $\pt{p}_3$, such that the triangle corresponds to the set
%
\begin{equation*}
  \sett{\tau}(\pt{p}_1, \pt{p}_2, \pt{p}_3) = \left\{ t_1\pt{p}_1 + t_2\pt{p}_2 + t_3\pt{p}_3 ~ \big| ~ t_1, t_2, t_3 \in \mathbb{R}_{\ge0}, \, t_1 + t_2+t_3 \le 1 \right\} \, \text{.}
\end{equation*}

\paragraph{\Disk{}}
A disk $\sett{\phi}$ is defined by a radius $r$, a center point $\pt{o}$, and a unit normal vector to the disk face $\et{n}$. Equivalently, using the same notation for the center point and the unit normal vector to the face, a disk can be defined by a radius $r$ and a laying plane ${\pi}(\pt{o}, \et{n})$. In both cases, the disk corresponds to the set
%
\begin{equation*}
  \sett{\phi}(r, \pt{o}, \et{n}) = \sett{\phi}(r, {\pi}(\pt{o}, \et{n})) = \left\{ \big\| \pt{o} - \left[x, y, z\right]^{\top} \big\|^2_2 \le r^2 ~ \big| ~ \left[x, y, z\right]^{\top} \in {\pi}(\pt{o}, \et{n}) \right\} \, \text{.}
\end{equation*}

\paragraph{\Ball{}}
A ball $\sett{\omega}$ is defined by a radius $r$ and a center point $\pt{o}$, such that it corresponds to the set
%
\begin{equation*}
  \sett{\omega}(r, \pt{o}) = \left\{ \big\| \pt{o} - \left[x, y, z\right]^{\top} \big\|^2_2 \le r^2 ~ \big| ~\left[x, y, z\right]^{\top} \in \mathbb{R}^3 \right\} \, \text{.}
\end{equation*}

\subsection{Mesh Tools}
In addition to the fundamental data types presented, we also provide other classes that are useful in scenarios involving mesh or manipulation of large numbers of entities. These objects include \Collection{}, \ac{AABB}, and \AabbTree{}.

\paragraph{\Collection{}}
The \Collection{} object consists of a vector of \SharedPointer{} to \Entity{} type objects. This class can be used when a substantial number of \Entity{} object instances need to be grouped into a single object. The grouping, coupled with the usage of \SharedPointer{} objects, facilitates effective data manipulation and ensures safe memory management. Nonetheless, the \Collection{} object is not a geometric entity and does not have any geometric meaning. It is merely a container for \Entity{} objects that can be used to perform operations on a large number of \Entity{} objects simultaneously. In Figure~\ref{app1:fig:acme_collection}, an example of two \Collection{} objects bounded in two different \Aabb{}s is reported.

\begin{figure}[htb]
  \centering
  \def\svgwidth{8.0cm}
  \input{./figures/appendix_1/acme_collection.pdf_tex}
  \caption{Example \Collection{} objects bounded in two different \Aabb{}s. The two \Aabb{} objects are then bounded in a master \Aabb{} depicted in red.}
  \label{app1:fig:acme_collection}
\end{figure}

\paragraph{\Aabb{}}
An \Aabb{} $\sett{\beta}$ is defined by a maximum point $\pt{p}_{\max}$ and a minimum point $\pt{p}_{\min}$, which are respectively equal to
%
\begin{equation*}
  \pt{p}_{\max} = \left[x_{\max}, y_{\max}, z_{\max}\right]^{\top}
  \quad \text{and} \quad
  \pt{p}_{\min} = \left[x_{\min}, y_{\min}, z_{\min}\right]^{\top} \, \text{.}
\end{equation*}
%
The \Aabb{} corresponds to the set
%
\begin{equation*}
  \sett{\beta} (\pt{p}_{\max}, \pt{p}_{\min}) =
  \left\{
  \left[x, y, z\right]^{\top} \in \mathbb{R}^3 ~ \bigg| ~
  \begin{array}{c}
    x_{\min} \leq x \leq x_{\max} \\
    y_{\min} \leq y \leq y_{\max} \\
    z_{\min} \leq z \leq z_{\max}
  \end{array}
  \right\} \, \text{.}
\end{equation*}
%
Indeed, this type of geometrical entity is very simple, requiring only two \Point{} objects to fully describe the space it occupies. Furthermore, the algorithms involved in \Aabb{} collision detection and/or intersection are highly efficient. Specifically, the basic algorithm for \Aabb{}-\Aabb{} collision detection can be executed solely through two-way comparison operators, making it lightweight and fast to perform.

\paragraph{\AabbTree{}}
There are plenty of possible tree structures. Some of them are suitable for a more rough spatial description with low computational complexity, while others are suitable for accurate spatial indexing but carry high computational complexity. In the \Acme{} library, the \ac{AABB} tree is chosen due to its balanced complexity-performance ratio, making it effective for \ac{RT} applications. The performance of a generic \ac{BVH} is generally measured by the computation time required to solve an intersection query. To enhance the \ac{BVH} performance and consequently reduce the number of comparisons among pairs of \acp{BVP}, a \ac{BVH} should be as compact as possible, minimizing the bounding volume contained in each \ac{BVP}~\cite{asyrani2012bounding, eloe2014dual}. Several techniques can be employed to build an \ac{AABB} tree, with the most common being the \emph{top-down} and \emph{bottom-up} strategies. While the \emph{top-down} strategy allows to easily perform the tree construction~\cite{eloe2014dual, ericson2004realtime, asyrani2012bounding}, the \emph{bottom-up} approach usually achieves more compact trees and better performances, albeit being more intricate to construct~\cite{omohundro1989five, asyrani2012bounding}. The typical average computational complexity of tree construction is $\mathcal{O}(n\log{n})$, while the intersection of two \ac{AABB} trees has an average cost of $\mathcal{O}(m\log{n})$, where $n$ and $m$ are the numbers of \acp{BVP} of the two trees. If one intends to intersect two sets of \acp{BVP} without the use of the \ac{AABB} tree the computational cost is always $\mathcal{O}(nm)$, as each element of the first set must be compared with all the elements of the second set~\cite{xing2010efficient}. Notably, The \AabbTree{} implemented in the \Acme{} library is directly derived from the one presented in~\cite{frego2019pointcoloud, bertolazzi2020efficient} and has been extended from the 2D to the 3D case (please refer to~\cite{stocco2021acme} for a detailed description of the \AabbTree{} implementation).

\subsection{Basic Intersection Algorithms}

Specific algorithms for basic intersections are not discussed here for the sake of brevity. It is important to note that comprehensive sources for intersection testing are limited. Exceptions include~\cite{schneider2002geometric} and~\cite{eberly2020robust}, which serve as extensive collections of geometric tests of various types. References~\cite{ericson2004realtime} and~\cite{vandenbergen2003collision} are equally valuable, although not as exhaustive. Individual articles on specific tests can also be found in the five-volume \emph{Graphic Gems} series~\cite{glassner1990graphics, arvo1991graphics, kirk1992graphics, heckbert1994graphics, paeth1995graphics}.

\subsection{Software Functionalities}
The \Acme{} geometry library consists of a \cpp{} core and a \Matlab{} \Mex{} wrapper. The library is built to efficiently \emph{create}, \emph{intersect} and \emph{destroy} basic geometry entities objects. It is possible to check geometrical conditions between objects, like \emph{parallelism}, \emph{orthogonality}, \emph{collinearity} and \emph{coplanarity}. The intersections that can be performed with the \Acme{} library are limited to those that potentially return a \emph{single} \Acme{}::\Entity{} object. For example, the intersection of a coplanar disk and a triangle may potentially return a circular arc and two segments, making it unsuitable for direct execution through the \Acme{} library. The sets of geometrical condition tests and intersections that can be performed are summarized in Tables~\ref{app1:tab:acme_conditions} and~\ref{app1:tab:acme_intersections} respectively.

\begin{table}[htb]
  \centering
  \setlength{\tabcolsep}{0.3em}
  \begin{tabular}{ccccccccc}
    \toprule
    \makecell[cc]{\textbf{Geometrical}\\\textbf{intersection}\\\textbf{tests}} &
    \rotatebox[origin=c]{270}{~~\Point{}~~}    &
    \rotatebox[origin=c]{270}{~~\Line{}~~}     &
    \rotatebox[origin=c]{270}{~~\Ray{}~~}      &
    \rotatebox[origin=c]{270}{~~\Plane{}~~}    &
    \rotatebox[origin=c]{270}{~~\Segment{}~~}  &
    \rotatebox[origin=c]{270}{~~\Triangle{}~~} &
    \rotatebox[origin=c]{270}{~~\Disk{}~~}     &
    \rotatebox[origin=c]{270}{~~\Ball{}~~}     \\
    \midrule
    Parallelism   & $-$ & $\bullet$ & $\bullet$ & $\bullet$ & $\bullet$ & $\bullet$ & $\bullet$ & $-$ \\
    Orthogonality & $-$ & $\bullet$ & $\bullet$ & $\bullet$ & $\bullet$ & $\bullet$ & $\bullet$ & $-$ \\
    Collinearity  & $-$ & $\bullet$ & $\bullet$ & $-$       & $-$       & $-$       & $-$       & $-$ \\
    Coplanarity   & $-$ & $\bullet$ & $\bullet$ & $\bullet$ & $\bullet$ & $\bullet$ & $\bullet$ & $-$ \\
    \bottomrule
  \end{tabular}
  \caption{Geometrical conditions tests that can be performed through \Acme{} library. \emph{Legend}: $\bullet$ available test, and $-$ not available test.}
  \label{app1:tab:acme_conditions}
\end{table}

\begin{table}[htb]
  \centering
  \setlength{\tabcolsep}{0.3em}
  \begin{tabular}{ccccccccc}
    \toprule
    \makecell[cc]{\textbf{Geometrical}\\\textbf{intersection}\\\textbf{tests}} &
    \rotatebox[origin=c]{270}{~~\Point{}~~}    &
    \rotatebox[origin=c]{270}{~~\Line{}~~}     &
    \rotatebox[origin=c]{270}{~~\Ray{}~~}      &
    \rotatebox[origin=c]{270}{~~\Plane{}~~}    &
    \rotatebox[origin=c]{270}{~~\Segment{}~~}  &
    \rotatebox[origin=c]{270}{~~\Triangle{}~~} &
    \rotatebox[origin=c]{270}{~~\Disk{}~~}     &
    \rotatebox[origin=c]{270}{~~\Ball{}~~}     \\
    \midrule
    \Point{}    & $\bullet$ & $\bullet$ & $\bullet$ & $\bullet$ & $\bullet$ & $\bullet$ & $\bullet$ & $\bullet$ \\
    \Line{}     & $\bullet$ & $\bullet$ & $\bullet$ & $\bullet$ & $\bullet$ & $\bullet$ & $\bullet$ & $\bullet$ \\
    \Ray{}      & $\bullet$ & $\bullet$ & $\bullet$ & $\bullet$ & $\bullet$ & $\bullet$ & $\bullet$ & $\bullet$ \\
    \Plane{}    & $\bullet$ & $\bullet$ & $\bullet$ & $\bullet$ & $\bullet$ & $\bullet$ & $\bullet$ & $\bullet$ \\
    \Segment{}  & $\bullet$ & $\bullet$ & $\bullet$ & $\bullet$ & $\bullet$ & $\bullet$ & $\bullet$ & $\bullet$ \\
    \Triangle{} & $\bullet$ & $\bullet$ & $\bullet$ & $\bullet$ & $\bullet$ & $\circ$   & $\circ$   & $-$       \\
    \Disk{}     & $\bullet$ & $\bullet$ & $\bullet$ & $\bullet$ & $\bullet$ & $\circ$   & $\circ$   & $-$       \\
    \Ball{}     & $\bullet$ & $\bullet$ & $\bullet$ & $\bullet$ & $\bullet$ & $-$       & $-$       & $-$       \\
    \bottomrule
  \end{tabular}
  \caption{Geometrical intersection tests that can be performed through \Acme{} library. \emph{Legend}: $\bullet$ intersection can be always performed, $\circ$ intersection can be performed only if entities are not coplanar, and $-$ intersection can not be performed.}
  \label{app1:tab:acme_intersections}
\end{table}

Thanks to the \Matlab{} \Mex{}, objects can also be manipulated and \emph{visualized} in the \Matlab{} environment. An interesting feature of the \Matlab{} \Mex{} is that it preserves \cpp{} polymorphism. In other words, when performing an intersection between two generic objects, both in \cpp{} and in \Matlab{}, the software outputs the exact data type of the entity resulting from the intersection, maintaining all checks and verifications transparent to the end-user.

Table~\ref{app1:tab:acme_timing} presents a comparison of timing performances between the \CGAL{} and \Acme{} libraries in a \cpp{} environment. As evident from the results, there is a notable increase in speed. This could be attributed to the greater complexity of the \CGAL{} library, which, in addition to having a much more intricate and comprehensive framework than \Acme{}, likely carries out additional checks or dynamic allocations on the objects in use.


\begin{table}[htb]
  \centering
  \begin{tabular}{cccccc}
    \toprule
    \multirow{2.5}{*}{\makecell[cc]{\textbf{Intersected}\\\textbf{entities}}} &
    \multicolumn{2}{c}{\textbf{\CGAL{}}} & \multicolumn{2}{c}{\textbf{\Acme{}}} &
    \textbf{Speed-up} \\ \cmidrule(l{4pt}r{4pt}){2-3} \cmidrule(l{4pt}r{4pt}){4-5}
    & $\mu$~(\USI{\nano\second}) & $\sigma^2$~(\USI{\nano\second\squared}) &
      $\mu$~(\USI{\nano\second}) & $\sigma^2$~(\USI{\nano\second\squared}) &
      ($\times$) \\
    \midrule
    \Line{}-\Line{}         & \num{18.3} & \num{0.291}  & \num{2.2}  & \num{0.0132} & \num{8.3} \\
    \Ray{}-\Ray{}           & \num{1030} & \num{0.739}  & \num{8.5}  & \num{0.0974} & \num{121} \\
    \Segment{}-\Segment{}   & \num{1050} & \num{1.05}   & \num{8}    & \num{0.138}  & \num{131} \\
    \Triangle{}-\Triangle{} & \num{2920} & \num{3.83}   & \num{24}   & \num{0.454}  & \num{121} \\
    \Line{}-\Ray{}          & \num{13}   & \num{0.0268} & \num{1.9}  & \num{0.0164} & \num{6.8} \\
    \Line{}-\Segment{}      & \num{17}   & \num{0.104}  & \num{6.9}  & \num{0.192}  & \num{2.4} \\
    \Line{}-\Triangle{}     & \num{11.3} & \num{0.0228} & \num{5.6}  & \num{0.0526} & \num{2} \\
    \Ray{}-\Triangle{}      & \num{11.6} & \num{2.28}   & \num{25.5} & \num{0.858}  & \num{-0.45} \\
    \Segment{}-\Triangle{}  & \num{15}   & \num{0.568}  & \num{13.8} & \num{0.247}  & \num{1.1} \\
    \bottomrule
  \end{tabular}
  \caption{Timing performance comparison between \CGAL{} and \Acme{} libraries. The test consists of $10^{5}$ intersections between randomly created objects. Notice that intersections are only made between types of geometric entities common to the two libraries. \emph{Legend}: $\mu$ average intersection run-time, and $\sigma^2$ intersection run-time variance.}
  \label{app1:tab:acme_timing}
\end{table}

% % % % % % % % % % % % % % % % % % % % % % % % % % % % % % % % % % % % % % % %

\section{A Step-by-Step Example}

\begin{figure}[htb]
  \centering
  \small{\includetikz{figures/appendix_1/acme_example.tex}}
  \caption{Visualization of the example problem, which is obtained through the last 6 lines of the \Matlab{} example code.}
  \label{app1:fig:acme_example}
\end{figure}

We now present an example that illustrates some capabilities of the \Acme{} library. Specifically, the same example will be presented in both the \cpp{} language and the \Matlab{} environment, allowing us to understand the few differences between the two working environments. In the following \cpp{} and \Matlab{} code snippets, we will create the \Disk{} objects
%
\begin{equation*}
  \sett{\phi}_1(r_1, \pt{o}_1, \et{n}_1) = \sett{\phi}_1(2, \, \left[0, 0, 0\right]^{\top}, \left[0, 1, 0\right]^{\top}) \, \text{,}
\end{equation*}
%
and
%
\begin{equation*}
\sett{\phi}_2(r_2, \pt{o}_2, \et{n}_2) = \sett{\phi}_2(1, \, \left[0, 0, 0\right]^{\top}, \left[1, 1, 0\right]^{\top}) \, \text{,}
\end{equation*}
%
that will be indicated by the variables \code{d1} and \code{d2}, respectively. Then, we will then intersect them, obtaining a geometric entity whose type is unknown to us. Subsequently, we will use the \code{type()} method to identify and print a string describing the type of the obtained \Entity{} object. In both cases, the output will be the string ``\code{segment}''. Finally, we will plot the obtained \Entity{} object in a \Matlab{} figure. The \cpp{} and \Matlab{} code snippets are reported in the following. The visualization of the example problem is reported in Figure~\ref{app1:fig:acme_example}, which is obtained through the last 6 lines of the \Matlab{} example code. \\[1.0em]

\begin{minipage}[t]{0.475\textwidth}
\cpp{}
\begin{mapleboxed}
#include "acme.hh"
using namespace acme;
using namespace std;

int main(void){
  // Create the disks
  entity *d1 = new disk(
    2, point(0,0,0), vec3(0,1,0)
  );
  entity *d2 = new disk(
    1, point(0,0,0), vec3(1,1,0)
  );

  // Perform the intersection
  entity *e1 = intersection(d1,d2);

  // Check output entity type
  cout << e1->type() << endl;
  return 0;
}
\end{mapleboxed}
\end{minipage}
\hfill
\begin{minipage}[t]{0.475\textwidth}
\Matlab{}
\begin{mapleboxed}
% Create the disks
d1 = acme_disk( ...
  2, [0,0,0]', [0,1,0]' ...
);
d2 = acme_disk( ...
  1, [0,0,0]', [1,1,0]' ...
);

% Perform the intersection
e1 = d1.intersection(d2);

% Check output entity type
disp(e1.type());

% Plot output
f1 = figure;
grid on; grid minor;
d1.plot(f1, 'red');
d2.plot(f1, 'blue');
e1.plot(f1, 'green');
\end{mapleboxed}
\end{minipage}

% % % % % % % % % % % % % % % % % % % % % % % % % % % % % % % % % % % % % % % %
