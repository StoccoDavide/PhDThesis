%!TEX root = ../main.tex

\chapter{Symbolic Linear Algebra and Applications}
\label{chapter:symbolic_linear_algebra}


Capitolo dove si descrive LEM e LAST perche si usa nelle DAE (riduzione indice) e TRUSS ME per superare limiti di MAPLE.

La fattorizzazione simbolica di una matrice è un processo che consiste nel rappresentare una matrice come prodotto di altre matrici più semplici o speciali. Mentre il calcolo simbolico è spesso associato alla manipolazione di espressioni matematiche, in alcuni contesti può essere utile estenderlo all'analisi numerica, in particolare quando si tratta di risolvere sistemi di equazioni lineari.

Ecco alcune ragioni per cui la fattorizzazione simbolica di una matrice può essere importante in analisi numerica:

Precisione nelle Operazioni Numeriche:
La fattorizzazione simbolica può fornire espressioni più semplici e precise per le matrici coinvolte in un problema numerico. Questo può ridurre la propagazione degli errori di arrotondamento durante le operazioni numeriche, migliorando così la precisione della soluzione numerica.
Ottimizzazione degli Algoritmi Numerici:
La conoscenza della struttura simbolica della matrice può consentire di progettare algoritmi numerici più efficienti. Ad esempio, si possono identificare sottoproblemi che possono essere risolti più rapidamente sfruttando la struttura particolare della matrice.
Analisi della Stabilità Numerica:
Nel contesto della risoluzione di sistemi lineari, la fattorizzazione simbolica può essere utilizzata per studiare la stabilità numerica degli algoritmi. Essa fornisce informazioni sulle caratteristiche strutturali della matrice che possono influenzare la stabilità degli algoritmi utilizzati.
Ottimizzazione della Complessità Computazionale:
La fattorizzazione simbolica può semplificare le espressioni coinvolte nei calcoli numerici, riducendo la complessità computazionale e migliorando l'efficienza degli algoritmi. Ciò è particolarmente rilevante quando si lavora con matrici sparse o con una struttura particolare.
Risoluzione Rapida di Sistemi Lineari Successivi:
In alcuni casi, la fattorizzazione simbolica può essere utilizzata per risolvere rapidamente sistemi lineari successivi con la stessa matrice coefficiente. Poiché la fattorizzazione può essere riutilizzata, si evita di ripetere costose operazioni di decomposizione numerica.
Comprendere la Struttura del Problema:
La rappresentazione simbolica della matrice fornisce una visione più chiara della struttura matematica del problema, agevolando la comprensione dei pattern e delle relazioni matematiche sottostanti.

In sintesi, la fattorizzazione simbolica di una matrice può essere un utile strumento in analisi numerica per migliorare la precisione, l'efficienza e la stabilità degli algoritmi utilizzati nella risoluzione di sistemi lineari e in altri contesti matriciali.
