%!TEX root = ../main.tex

\chapter{Conclusions}
\label{chap:conclusions}

In the previous chapter, we presented a methodology for the automatic index reduction of \ac{DAE} systems. The index reduction algorithm is based on the separation of the system into differential and algebraic parts with the help of symbolic linear algebra, \ie{}, \ac{LU} or \ac{FFLU} matrix factorizations. Symbolic-numerical examples are presented to detail the capabilities of the proposed index reduction algorithm. The results show that the presented algorithm is capable of consistently reducing high-index \ac{DAE} systems to index-0 \acp{DAE}. The computational cost of the expressions generated during the index reduction procedure is comparable to the original \ac{DAE} system in most of the examples. However, the \Maple{} symbolic computation kernel is not always able to perform the simplification in some cases. As a consequence, the expression complexity increases significantly throughout the left reduction steps. Still, the presented algorithm can successfully reduce the index of the \ac{DAE} system. Yet, the numerical efficiency of the final \ac{DAE} system is undermined by the inherent increase in computational complexity of the expressions generated in the last reduction steps. This highlights the need for future research on the inclusion of large expression management techniques in the symbolic index reduction algorithm to limit expression swell, as well as to augment the \ac{DAE} system and obtain a more compact representation of the expressions generated during the index reduction procedure. Despite this, the reduced-index systems are proven to retain good numerical stability during the integration process. A comparison between the joint index reduction algorithm and numerical integration schemes offered by \Maple{} with those of \Indigo{} demonstrates the effectiveness of the proposed methodology and software implementation.
