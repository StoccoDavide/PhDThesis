%!TEX root = ../main.tex

\chapter{Introduction}
\label{chap1:introduction}

% % % % % % % % % % % % % % % % % % % % % % % % % % % % % % % % % % % % % % % %

\section{Research Objectives and Contributions}

Notably, in findings presented by the group of Roswitha M{\"a}rz, which are summarized in~\cite{lamour2013differential}, there is little mention of symbolic computation tools in the derivation of projector functions and matrix function sequences. Instead, the authors rely on numerical factorization methods with a coupled automatic differentiation method to compute the projector functions and the matrix function sequences~\cite{lamour2011computational, schwarz2015diagnosis}. However, it is noteworthy that in \citet{lamour2013differential}, authors show that the latest state-of-the-art \ac{CAS} allows the derivation of smooth symbolic projector functions, especially for simple or small-scale \acp{DAE}. This can significantly reduce the computational cost of the projector-based analysis method as it eliminates the need for numerical factorization methods and automatic differentiation.

Nevertheless, the most relevant consideration arising from what we stated in Section~\ref{chap1:sec:index_reduction_methods}, as well as in \citet{bojarincev1980regular, gear1984ode, griepentrog1989basic} (despite being these works focused on linear \acp{DAE}), is that a neat separation between the differential and algebraic equations of the \ac{DAE} system is crucial for the successful application of the differential index reduction method. This separation is achieved by transforming the \ac{DAE} system into its Hessenberg form. However, this process is not always straightforward. Particularly, for generic nonlinear \acp{DAE}, the separation process can be computationally expensive and time-consuming.

Starting from these observations, the primary objective of this thesis is to develop an index reduction algorithm for \acp{DAE} based on symbolic matrix factorization techniques. It is worth noting that the algorithm is neither directly based on the projector-based analysis nor in index concepts other than the differential index. The algorithm is limited to generic well-determined \acp{DAE} of the form~\eqref{chap1:eq:dae}, linear in the states' derivatives. The specific research objectives and contributions to the current state-of-the-art of this thesis are listed below.
%
\begin{itemize}
  \item The development of a symbolic linear algebra package for symbolically solving linear systems of equations using matrix factorization techniques. The package, including hierarchical representation techniques for expression swell mitigation and large expression management, enables the \Maple{} kernel to handle larger-scale linear systems of equations. This seemingly simple contribution is crucial for the subsequent development of the symbolic index reduction algorithm for \acp{DAE} as it boosts the computational efficiency of the involved symbolic operations.
  \item The development of a novel symbolic index reduction algorithm for \acp{DAE} based on the symbolic matrix factorization techniques introduced in the previous objective. Specifically, the symbolic linear algebra package is employed to separate the differential and algebraic equations of the \ac{DAE} system. A subsequent differentiation is applied to the algebraic equations to reduce the index of the \ac{DAE} system.
  \item The exploitation of the expressions' hierarchical representation structure, which is employed to mitigate the expression swelling, as a set of index-1 variables in a newly developed numerical scheme for integrating reduced-index \acp{DAE}.
  \item Lastly, the validation of the symbolic index reduction algorithm through a set of benchmark \acp{DAE} from the literature arising in various fields, as well as the comparison of the algorithm's performance with the \Maple{}'s built-in pure symbolic-based index reduction algorithm.
\end{itemize}
%
Nonetheless, the symbolic algorithm is implemented as an open-source \Maple{} package. On the other hand, the numerical scheme is included in an open-source \Matlab{} toolbox. Both software are collected in the \Indigo{} toolbox~\cite{indigo}, whose dependencies are the symbolic linear algebra package \LAST{}~\cite{last} and the large expression management package \LEM{}~\cite{lem}. All the software presented in this thesis is distributed under the \ac{BSD} 3-Clause License, which allows for both academic and commercial use. The proposed symbolic algorithm and its dedicated numerical scheme are validated through a set of benchmark \acp{DAE} from the literature arising in various fields, including problems from \ac{MBD}, electrical circuit simulation, and \ac{TPPC}.

% % % % % % % % % % % % % % % % % % % % % % % % % % % % % % % % % % % % % % % %

\section{Thesis Outline}

After this introductory chapter, the remainder of this thesis is organized as follows. Chapter~\ref{chap3:symbolic_computation} provides a brief overview of the current state-of-the-art symbolic computation techniques, applications, and tools for solving foundational problems in mathematics, physics, and engineering. Then, it introduces the expression swell phenomenon and presents a symbolic computation technique used to mitigate it. The last part of the chapter is dedicated to the development of a computation package for solving large-scale linear systems of equations through symbolic linear algebra techniques, which will be used in the subsequent chapter to tackle \ac{DAE} systems. Chapter~\ref{chap4:daes} is mainly dedicated to the development of a symbolic index reduction algorithm for \acp{DAE} based on the previously mentioned symbolic matrix factorization techniques. It presents the index reduction algorithm implementation, which is written in the \Maple{} language. Lastly, the numerical scheme utilized to integrate the reduced-index \acp{DAE} is validated through an index-3 \acp{DAE} example. Chapter~\ref{chap5:applications} demonstrates the application of the symbolic index reduction algorithm to a set of benchmark \acp{DAE} from the literature arising in various fields, including \ac{MBD}, electrical circuit simulation, and \ac{TPPC}. The chapter concludes with a summary of the presented algorithm performance. Finally, Chapter~\ref{chap6:conclusions} summarizes the main contributions of this thesis and discusses possible future developments.

Appendices are included at the end of the thesis to provide additional information on side topics that are not essential to the main topic of this thesis but are relevant for one of the benchmark \acp{DAE} presented in Chapter~\ref{chap5:applications}. The appendices include the presentation of a \cpp{} computational geometry library (Appendix~\ref{app1:acme}), as well as the derivation of models for tire-ground enveloping (Appendix~\ref{app2:enve}), tire contact forces (Appendix~\ref{app3:tirex}). The last appendix (Appendix~\ref{app4:trussme}) provides an overview of the development of a symbolic package for solving structures using the \ac{DSM}.

% % % % % % % % % % % % % % % % % % % % % % % % % % % % % % % % % % % % % % % %
