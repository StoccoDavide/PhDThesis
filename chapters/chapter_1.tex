%!TEX root = ../main.tex

\chapter{Introduction}
\label{chapter:introduction}

\section{Bla bla}
\subsection{Bla bla}
\subsubsection{Bla bla}

L'analisi numerica è una disciplina che si occupa dello sviluppo e dell'applicazione di metodi computazionali per risolvere problemi matematici attraverso l'uso di algoritmi e tecniche numeriche. Le sue origini risalgono ai tempi antichi, quando le prime società umane iniziarono a sviluppare metodi approssimati per risolvere equazioni e misurare grandezze. Tuttavia, il vero impulso per l'analisi numerica si è verificato con l'avvento delle macchine calcolatrici e dei computer nel XX secolo.

Nel corso del tempo, con l'aumentare della complessità dei problemi scientifici e ingegneristici, è diventato evidente che era necessario sviluppare approcci più sofisticati per risolvere equazioni differenziali, integrali e altri problemi matematici. Il lavoro pionieristico di matematici come John von Neumann e Alonzo Church ha contribuito allo sviluppo di metodi numerici fondamentali, tra cui l'approssimazione di funzioni, l'integrazione numerica e la risoluzione di sistemi di equazioni lineari e non lineari.

Con l'avanzare della tecnologia informatica, è emersa una nuova branca dell'analisi numerica nota come calcolo simbolico. Questo approccio si distingue per la manipolazione simbolica delle espressioni matematiche, consentendo la rappresentazione e la manipolazione di variabili simboliche anziché numeriche. Il calcolo simbolico ha trovato applicazioni in diverse discipline, dalla matematica pura all'ingegneria, fornendo strumenti potenti per la semplificazione di formule, la derivazione simbolica e la risoluzione simbolica di equazioni.

Il calcolo simbolico e l'analisi numerica, dunque, collaborano per offrire soluzioni complete a una vasta gamma di problemi matematici e scientifici. Mentre l'analisi numerica si concentra sull'approccio computazionale e approssimativo, il calcolo simbolico offre precisione attraverso la manipolazione simbolica delle espressioni matematiche, portando a una combinazione sinergica di potenti strumenti per l'indagine matematica e scientifica.