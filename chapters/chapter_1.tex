%!TEX root = ../main.tex

\chapter{Introduction}
\label{chap1:introduction}

The solution of dynamical systems represents a fundamental challenge in fields like applied mathematics, engineering, and the physical sciences. These systems, serving as mathematical representations, describe how a system evolves, offering insights into phenomena ranging from chemical reactions to electrical circuit dynamics. Resolving dynamical systems involves determining the system's state at each time point, considering its \acp{IC} and governing equations of motion. This process is critical for comprehending complex system behavior and predicting its evolution. However, due to their inherent complexity, solving such dynamical systems often requires sophisticated algorithms and high-performance computing resources.

As a matter of fact, while the solution of \acl{ODE} systems has been extensively studied and is well understood, the solution of \acl{DAE} systems is more challenging due to the presence of algebraic constraints. Specifically, \acp{DAE} are a generalization of \acp{ODE} that involve both differential and algebraic equations. They first appeared in the last century to model mechanisms and, over these decades, they gained popularity in various fields. Indeed, \acp{DAE} are a powerful tool for modeling systems with constraints~\cite{burger2018dae}, like mechanical systems, electrical circuits, and chemical processes, due to their simplicity- and straightforwardness-of-use. However, conversely to \acp{ODE}, the solution of \acp{DAE} requires specialized algorithms to be solved accurately and efficiently. In this chapter, we offer an overview of \ac{DAE} system and their state-of-the-art solution methods. Afterward, we clarify the motivation behind this research and the contributions of this thesis: The development of a novel and robust algorithm for the index reduction of generic first-order \acp{DAE} using symbolic computation methods.

% % % % % % % % % % % % % % % % % % % % % % % % % % % % % % % % % % % % % % % %

\section{Differential-Algebraic Equations}

A system of equations involving one or more unknown functions and their derivatives is referred to as a \ac{DAE} system. In its general form, a first-order \ac{DAE} system is expressed as
%
\begin{equation}
  \m{F}(\m{x}, \m{x}^{\prime}, t) = \m{0} \, \text{,}
  \label{chap1:eq:dae}
\end{equation}
%
where $\m{x}(t) = \m{x}$ denotes the vector of unknown functions, and $\m{F}(\m{x}, \m{x}^{\prime}, t) = \m{F}$ consists of $n$ components. Notice the components of $\m{x}$ and $\m{F}$ are denoted as $x_i$ and $F_i$, respectively, for $i = 1, 2, \dots, n$. The term ``\ac{DAE}'' is typically used when the highest derivative $\m{x}^{\prime}$ can not be explicitly solved in terms of the other variables $\m{x}$ and $t$ within the algebraic relationship represented by~\eqref{chap1:eq:dae}. Notably, the Jacobian $\jac{\m{F}}{\m{x}^{\prime}}$ along a specific solution of the \ac{DAE} might become singular. Systems of equations like~\eqref{chap1:eq:dae} are also known as \emph{implicit} systems. Depending on the context, a \ac{DAE} may represent either an \ac{IVP}, where $\m{x}$ is specified at the initial time (\eg{}, $x(t_0) = x_0$), or a \ac{BVP}, where the solution adheres to $n$ two-point \acp{BC} (\eg{}, $g(x(t_0), x(t_f)) = 0$).

The solution approach for a \ac{DAE} depends on its structural characteristics. A prevalent category of \acp{DAE} encountered in practical applications is the semi-explicit \ac{DAE}, represented as
%
\begin{subequations}
  \begin{empheq}[left = {\empheqlbrace}, right = {\, \text{,}}]{align}
    & \m{x}^{\prime} = \m{f}(\m{x}, \m{z}, t) \label{chap1:eq:semiexplicit_dae_diff} \\
    & \m{0} = \m{g}(\m{x}, \m{z}, t) \label{chap1:eq:semiexplicit_dae_alg}
  \end{empheq}
  \label{chap1:eq:semiexplicit_dae}% not to indent the following text
\end{subequations}
%
where~\eqref{chap1:eq:semiexplicit_dae_diff} and~\eqref{chap1:eq:semiexplicit_dae_alg} denote the \emph{differential equations} and the \emph{algebraic constraints}, respectively. An example of a \ac{DAE} system of the form~\eqref{chap1:eq:semiexplicit_dae} is the simple pendulum in redundant coordinates, where the motion of a mass at the end of a rod is constrained to a fixed length. In the Cartesian coordinates, the system is described by a set of \acp{DAE} that can be transformed into the classical \ac{ODE} for a pendulum.
%
\begin{example}[The Simple Pendulum]
  Consider a straightforward example involving the motion of a pendulum in Cartesian coordinates (see Figure 1). Let the position and velocity coordinates of the mass $m$ at the end of the rod be $(x, y)$ and $(x^{\prime}, y^{\prime}) = (u, v)$ respectively, with the angle between the rod and the vertical axis denoted as $\theta$. The position of the mass is given by $x = \ell\sin{\theta}$ and $y = \ell\cos{\theta}$, where $\ell$ represents the length of the pendulum. The Euler-Lagrange equations yield the following first-order semi-explicit \ac{DAE} system
  %
  \begin{subequations}
    \begin{empheq}[left = {\empheqlbrace}, right = {\, \text{,}}]{align}
      & x^{\prime} = u \\
      & y^{\prime} = v \\
      & u^{\prime} = -\dfrac{2 \lambda x}{m} \\
      & v^{\prime} = -\dfrac{2 \lambda y}{m} - g \\
      & 0 = x^2 + y^2 - \ell^2 \label{chap1:eq:pendulum_dae_alg}
    \end{empheq}
    \label{chap1:eq:pendulum_dae}% not to indent the following text
  \end{subequations}
  %
  which corresponds to the form~\eqref{chap1:eq:semiexplicit_dae}. Here, $g$ denotes gravity, $\lambda$ is the Lagrange multiplier, and $\ell$ represents the length of the pendulum. The terms $2 \lambda x$ and $2 \lambda y$ represent the constraint force maintaining the constraint~\eqref{chap1:eq:pendulum_dae_alg} and thereby ensuring the rod's fixed length. In this simple case, transforming variables $x = \ell\sin{\theta}$ and $y = \ell\cos{\theta}$, followed by algebraic manipulation, leads to the classical \ac{ODE} for a pendulum $\theta^{\prime\prime} = -g\sin{\theta}$. However, in more complex scenarios, such simplifications may not be feasible.
\end{example}
%
\acp{DAE}, in both their general and specific formulations, are commonly encountered in mathematical models across engineering and scientific disciplines. Further real-life examples of \ac{DAE} systems, spanning from \ac{MB} mechanical systems to electrical circuits and \acp{TPPC}, are detailed in~\cite{brenan1995numerical}. It is important to note that while constraints in mechanical systems like the pendulum physically represent the system's limitations, those in other scenarios, such as \acp{TPPC}, may also serve as performance specifications or control objectives.

\subsection{The Significance of Differential-Algebraic Equations}

As mentioned previously, \ac{DAE} systems represent a generalization of \acp{ODE}, which can typically expressed as $\m{x}^{\prime} = \m{f}(\m{x}, t)$, and have a well-established literature in both mathematical theory and numerical solution. While \acp{ODE} can be viewed as a subset of \acp{DAE}, the broader scope of \acp{DAE} encompasses problems with distinct mathematical properties and presents unique challenges for numerical resolution. Implicit \ac{DAE} models, as demonstrated in the example above, often provide a more natural representation compared to explicit formulations. Consequently, working with \ac{DAE} models is highly advantageous, particularly when dealing with complex systems.

To grasp the distinction between \acp{DAE} and \acp{ODE}, let us consider the simple example
%
\begin{subequations}
  \begin{empheq}[left = {\empheqlbrace}, right = {\, \text{,}}]{align}
  & x^{\prime} = y \label{chap1:eq:dae_example_1} \\
  & 0 = x - q \label{chap1:eq:dae_example_2}
  \end{empheq}
  \label{chap1:eq:dae_example}% not to indent the following text
\end{subequations}
%
where $q = q(t)$ is a suitably smooth function. Clearly, the unique solution is $x = q$ and $y = q^{\prime}$, without the need for initial or \acp{BC}. This implies that imposing arbitrary \acp{IC} could render the \ac{DAE} inconsistent. Additionally, unlike \acp{ODE}, the solution's dependency on the derivative of the inhomogeneous part introduces a distinctive characteristic and, even with consistent initial values, the existence and uniqueness theory for \acp{DAE} involve more intricate technical assumptions beyond mere smoothness, as seen in the \ac{ODE} case. The requirement to differentiate $x$ in~\eqref{chap1:eq:dae_example_2}, thereby involving differentiation of the input function $q$ to determine $y$, constitutes a fundamental distinction. Hence, conversely to \acp{ODE}, \acp{DAE} may necessitate both integrations and differentiations to be resolved.

\subsection{The Index... Or Better Yet, the Indices}

In the realm of \acp{DAE}, the concept of index serves as a metric for measuring the deviation of a \ac{DAE} from its \emph{underlying} \ac{ODE} with \emph{invariants}. This index is a non-negative integer that provides valuable insights into the mathematical structure and potential complexities associated with analyzing and numerically solving the \ac{DAE}. Generally, a higher index indicates increased challenges in numerical resolution. Various index definitions exist, these includes the \emph{Kronecker} index (for linear constant coefficient \acp{DAE}), \emph{differentiation} index, \emph{structural} index, \emph{tractability} index, \emph{strangeness} index, \emph{geometric} index, and \emph{perturbation} index~\cite{mehrmann2015index}. While these indices may coincide in simpler scenarios, they can differ in more intricate nonlinear and fully implicit systems~\cite{lamour2012detecting}. Furthermore, the index may exhibit local variability, assuming different values across distinct regions and could even remain undefined at singular points~\cite{lamour2012detecting}. Notice that the differentiation index is the most common index used in practice and is typically referred to without any further epithet, \ie{}, ``the'' index. In the following sections, we delve into the differentiation, tractability, and structural indices, which are the most used in practice. Together with their qualitative definition, this brief discussion aims at providing a comprehensive understanding of the mathematical structure of \acp{DAE}, as well as the challenges associated with their resolution. For this purpose, several considerations, which arise from index definitions, are discussed in the following sections. It is important to notice that it is still a partially open problem to characterize the exact relationship between the existing indices, and the work on this topic is still ongoing~\cite{mehrmann2015index}, for this reason, we will not delve into their relationships.

\paragraph{Differential Index}

Given that a \ac{DAE} encompasses both differential equations and constraints, a potential strategy involves repeatedly differentiating the constraint equations (in the semi-explicit \ac{DAE} systems of the form~\eqref{chap1:eq:semiexplicit_dae}) and substituting them into the differential ones. This process, called \emph{index reduction}, aims to transform the \ac{DAE} into an explicit \ac{ODE} system for all unknowns. The solutions of the \ac{DAE} correspond to those solutions of the resulting \ac{ODE} residing within a subset termed as the \emph{solution manifold}~\cite{rheinboldt1984differential}. The number of iterations required for this transformation is termed the differential index of the \ac{DAE}, with the underlying \acp{ODE} possessing an index of 0~\cite{mehrmann2015index}.

\begin{example}[Differential Index]
  Consider the following simple examples involving a given smooth function $y = y(t)$ and the unknown $x$. Then the scalar equation
  %
  \begin{equation}
    x = y
    \label{chap1:eq:dae_index_1}
  \end{equation}
  %
  constitutes a \ac{DAE} with differential index of 1, as it requires one differentiation to yield the \ac{ODE} $x^{\prime} = y^{\prime}(t)$. Similarly, for the system
  %
  \begin{equation}
    \begin{cases}
      x_1 = y \\
      x_2 = x_1^{\prime}
    \end{cases}
    \label{chap1:eq:dae_index_2}
  \end{equation}
  %
  differentiating the first equation leads to $x_2 = x_1^{\prime} = y^{\prime}$, and subsequent differentiation yields $x_2^{\prime} = x_1^{\prime\prime} = y^{\prime\prime}$. Hence, this system possesses a differential index of 2, necessitating two differentiations to obtain the explicit underlying \ac{ODE} system.
\end{example}

It is crucial to understand that, while $n$ initial or boundary value conditions are required to define the solution of a first-order \acp{ODE} of size $n$, the solution of the simple \acp{DAE} in the previous example is solely determined by the right-hand side, necessitating only one consistent \ac{IC}. General \ac{DAE} systems often include \ac{ODE} subsystems, resulting in $k \in [0, n]$ \acp{DOF} for the \ac{DAE} solution. However, determining which $k$ pieces of information are necessary to determine the solution can be challenging. Initial or \acp{BC} specified for the \ac{DAE} must be consistent, satisfying both the constraints and the \emph{hidden} constraints of the system, which represent the solution manifold~\cite{rheinboldt1984differential}. For instance, in the index-1 system~\eqref{chap1:eq:dae_index_1}, an \ac{IC} must adhere to $x_1(0) = y(0)$. In the case of the index-2 system~\eqref{chap1:eq:dae_index_2}, meeting the additional constraint $x_2(0) = y^{\prime}(0)$ is necessary, along with $x_1(0) = y(0)$.

As can be noticed, the differential index represents a clear indication of how far is a \ac{DAE} from an \ac{ODE} system. However, computing the differential index is not a trivial task, as it involves intensive manipulation of expressions. The differentiation process can lead to the generation of large expressions, which can be inevitably computationally expensive to handle. Such limitations are particularly evident in the context of high-index \acp{DAE}. To address this challenge, other indices have been introduced to provide additional insights into the mathematical structure of \acp{DAE} without the need for extensive symbolic computation.

Although the concept of the differentiation index is widely used, it has a major drawback, since it is not suited for over- and underdetermined systems. The reason for this is that it is based on a solvability concept that requires unique solvability. For this reason, the \emph{strangeness index} was introduced, which extends the differentiation index to over- and underdetermined systems. The difference between the differentiation index and the strangeness index is that the former aims at reformulating the given problem as an \ac{ODE} for uniquely solvable systems, whereas the latter aims at reformulating it as a \ac{DAE} with two parts, one part that states all constraints and another part that describes the dynamical behavior~\cite{mehrmann2015index}. Despite the strangeness index's advantages, it is not widely used in practice, and the work on this index is still ongoing. Nonetheless, we limit our study only to well-determined \ac{DAE} systems.

\paragraph{Structural Index}

The structural index was initially introduced in the linear constant coefficient scenario for combinatorial analysis. Given the linear \acp{DAE} $\m{E}\m{x}^{\prime} = \m{A}\m{x} + \m{f}(t)$, the parameter-dependent pencil $(\m{E}(\m{p}), \m{A}(\m{p}))$ is constructed by substituting the nonzero elements of $\m{E}$ and $\m{A}$ with independent parameters $\m{p}$. The structural index, as discussed in~\cite{pantelides1988consistent, pryce2001simple} and expanded upon in subsequent works~\cite{benveniste2021structural}, refers to the unique integer value matching the \emph{Kronecker} index of $(\m{E}(\m{p}), \m{A}(\m{p}))$ across an open and dense subset of the parameter set. In the context of nonlinear systems, local linearization techniques are commonly applied. While it has been demonstrated in~\cite{reissig2000differential} that the differentiation index and the structural index may differ, Pantelides' algorithm, detailed in~\cite{pantelides1988consistent}, remains widely used in practical applications. Notably, in studies like~\cite{unger1995structural}, combinatorial insights are leveraged to determine which equations should be differentiated, as well as to introduce additional variables for index reduction~\cite{mattsson1993index}. However, a comprehensive assessment of the validity of this approach across various scenarios has been provided only in a few specific cases, as discussed in~\cite{mehrmann2015index}.

\paragraph{Tractability Index}

It is known from relevant studies, such as~\cite{lamour2013differential}, that the index may vary depending on a specific solution. An example of this is the \ac{DAE} system illustrated in the subsequent example.

\begin{example}[Differential Index Dependency on Solution]
  Consider the \ac{DAE} system in semi-explicit form
  %
  \begin{equation*}
    \begin{cases}
    x_1^{\prime} = x_3 \\
    0 = x_2(1 - x_2) \\
    0 = x_1x_2 + x_3(1 - x_2) - t
    \end{cases} \, \text{.}
  \end{equation*}
  %
  The second equation admits two solutions: $x_2 = 0$ and $x_2 = 1$. If $x_2$ is continuous, the system does not transition between these values. For $x_2 = 0$, the system is semi-explicit with an index of 1, while for $x_2 = 1$, the index becomes 2. Unlike the index-1 scenario, no initial value of $x_1$ is necessary. Replacing the algebraic equation involving $x_2$ with $x_2^{\prime} = 0$, leads the index of the modified \ac{DAE} system to depend on the \ac{IC}. Specifically, if $x_2(0) = 1$, the index is 2; otherwise, it remains 1~\cite[Section 3.3]{lamour2013differential}.
\end{example}

The tractability index concept is specifically designed to tackle the challenges posed by high-index \acp{DAE} with multiple regular regions. It explicitly suggests that the domain of definition of the \ac{DAE} can be divided into maximal regular regions, each delimited by singular points. Within each regular region, the \ac{DAE} exhibits a consistent structure, which can be revealed using matrix function sequences formed with permissible projector functions. This construction is governed by constant rank conditions. Singular points arise when this construction process encounters difficulties. A smooth flow and reliable treatment are expected as long as the solution remains within a regular region. However, crossing or touching a boundary between regions may lead to significant singularities. Essentially, monitoring the structure involves computing an admissible matrix function sequence and monitoring the rank conditions to ensure the stability and reliability of the solution~\cite{lamour2011computational}. Thereby, the fundamental idea behind the tractability index concept involves the utilization of derivatives of projectors instead of derivative arrays. However, if the numerical computation of the projectors is adopted, challenges may arise in obtaining the derivatives~\cite{mehrmann2015index}.

\subsection{The Hessenberg Forms}

As will be better detailed in the forthcoming section, the implicit \ac{DAE} system~\eqref{chap1:eq:dae} can represent mathematically ill-defined problems, as well as scenarios where direct discretization methods fail. Thankfully, numerous high-index problems encountered in practical applications can be expressed as a more restrictive formulation composed of \acp{ODE} with algebraic constraints. Within such systems, both the algebraic and differential variables are explicitly separated and identified, even for higher-index \acp{DAE}, allowing for the elimination of all algebraic variables using either index reduction, numerical direct discretization, or a combination of both. These formulations, known as \emph{Hessenberg forms} of the \acp{DAE}~\cite{brenan1995numerical}, are detailed here below and will serve as the basis for further exploration into \ac{DAE} solutions in subsequent sections and chapters.
%
\begin{itemize}
  \setlength\itemsep{0.0em}
  \item The \emph{Hessenberg index-1} \ac{DAE} system is represented as
  %
  \begin{equation*}
    \begin{cases}
      \m{x}^{\prime} = \m{f}(\m{x}, \m{z}, t) \\
      \m{0} = \m{g}(\m{x}, \m{z}, t)
    \end{cases} \, \text{,}
  \end{equation*}
  %
  with $\m{x} \in \mathbb{R}^n$, $\m{z} \in \mathbb{R}^m$, $\m{f}: \mathbb{R}^{n} \times \mathbb{R}^{m} \times \mathbb{R} \rightarrow \mathbb{R}^{n}$, $\m{g}: \mathbb{R}^{n} \times \mathbb{R}^{m} \times \mathbb{R} \rightarrow \mathbb{R}^{m}$, and where the Jacobian $\jac{\m{g}}{\m{z}}$ is assumed non-singular for all $t$. This configuration closely resembles the semi-explicit index-1 \ac{DAE} system~\eqref{chap1:eq:semiexplicit_dae} discussed earlier. Semi-explicit index-1 \acp{DAE} shares similarities with implicit \acp{ODE}. While it is theoretically feasible to solve for $\m{z}$ in the algebraic equation (using the implicit function theorem) and then substitute it into the differential equation to derive the underlying \ac{ODE} in terms of $\m{x}$ (though uniqueness is not guaranteed), this approach is not universally recommended for numerical solutions due to potential ill-conditioning and stability issues. An example of a Hessenberg index-1 \ac{DAE} system is in the formulation of \acp{TPPC} problems~\cite{brenan1995numerical}.
  %
  \item The \emph{Hessenberg index-2} \ac{DAE} system is expressed as
  %
  \begin{equation*}
    \begin{cases}
      \m{x}^{\prime} = \m{f}(\m{x}, \m{z}, t) \\
      \m{0} = \m{g}(\m{x}, t)
    \end{cases} \, \text{,}
  \end{equation*}
  %
  with $\m{x} \in \mathbb{R}^n$, $\m{z} \in \mathbb{R}^m$, $\m{f}: \mathbb{R}^{n} \times \mathbb{R}^{m} \times \mathbb{R} \rightarrow \mathbb{R}^{n}$, $\m{g}: \mathbb{R}^{n} \times \mathbb{R}^{m} \times \mathbb{R} \rightarrow \mathbb{R}^{m}$, and where the product of Jacobians $\jac{\m{g}}{\m{y}} \, \jac{\m{f}}{\m{z}}$ is assumed non-singular for all $t$. It is important to observe that the algebraic variable $\m{z}$ does not appear in the second equation. This arrangement characterizes a pure index-2 \ac{DAE}, where all algebraic variables act as index-2 variables. An example of Hessenberg index-2 \acp{DAE}arises from the modeling of incompressible fluid flow through discretized Navier-Stokes equations~\cite{ascher1998computer}.
  %
  \item The \emph{Hessenberg index-3} \ac{DAE} system is formulated as
  %
  \begin{equation*}
    \begin{cases}
      \m{x}^\prime = \m{f}(\m{x}, \m{y}, \m{z}, t) \\
      \m{y}^\prime = \m{g}(\m{x}, \m{y}, t) \\
      \m{0}        = \m{h}(\m{y}, t)
    \end{cases} \, \text{,}
  \end{equation*}
  %
  with $\m{x} \in \mathbb{R}^n$, $\m{z} \in \mathbb{R}^m$, $\m{f}: \mathbb{R}^{n} \times \mathbb{R}^{m} \times \mathbb{R} \rightarrow \mathbb{R}^{n}$, $\m{g}: \mathbb{R}^{n} \times \mathbb{R}^{m} \times \mathbb{R} \rightarrow \mathbb{R}^{m}$, $\m{h}: \mathbb{R}^{m} \times \mathbb{R} \rightarrow \mathbb{R}^{m}$, and where the product $\jac{\m{h}}{\m{y}} \, \jac{\m{g}}{\m{x}} \, \jac{\m{f}}{\m{z}}$ is non-singular for all $t$. Determining the index of a Hessenberg \acp{DAE} follows a similar differentiation process to the general case. However, only algebraic constraints need to be differentiated~\cite{ascher1991projected}. These index-3 \ac{DAE} systems are frequently encountered in practical scenarios, notably in the fields of \ac{MBD}, \acp{TPPC}, and various engineering applications~\cite{ascher1998computer,brenan1995numerical}.
  %
  \item The \emph{Hessenberg index-($r+2$)} \ac{DAE} system is formulated as
  %
  \begin{equation*}
    \begin{cases}
      \m{x}_r^\prime = \m{f}_r(\m{x}_1, \m{x}_2, \dots, \m{x}_r, \m{y}, \m{z}, t) \\
      \vdots \\
      \m{x}_2^\prime = \m{f}_2(\m{x}_1, \m{x}_2, \m{x}_3, \m{y}, t) \\
      \m{x}_1^\prime = \m{f}_1(\m{x}_1, \m{x}_2, \m{y}, t) \\
      \m{y}^\prime = \m{g}(\m{x}_1, \m{y}, t) \\
      \m{0}        = \m{h}(\m{y}, t)
    \end{cases} \quad \text{for} \quad r = 1, 2, \dots, r \, \text{,}
  \end{equation*}
  %
  with $\m{x}_i \in \mathbb{R}^{n_i}$, $\m{y} \in \mathbb{R}^m$, $\m{z} \in \mathbb{R}^m$, $\m{f}_i: \mathbb{R}^{n_i} \times \mathbb{R}^{n_{i+1}} \times \dots \times \mathbb{R}^{n_r} \times \mathbb{R}^{m} \times \mathbb{R} \rightarrow \mathbb{R}^{n_i}$, $\m{g}: \mathbb{R}^{n_1} \times \mathbb{R}^{m} \times \mathbb{R} \rightarrow \mathbb{R}^{m}$, $\m{h}: \mathbb{R}^{m} \times \mathbb{R} \rightarrow \mathbb{R}^{m}$, and where the product $\jac{\m{h}}{\m{y}} \, \jac{\m{g}}{\m{x}_1} \, \jac{\m{f}}{1\m{x}_2} \dots \jac{\m{f}}{r-1\m{x}_r} \, \jac{\m{f}}{r\m{z}}$ is non-singular for all $t$~\cite[Section 3.5]{lamour2013differential}.
\end{itemize}

% % % % % % % % % % % % % % % % % % % % % % % % % % % % % % % % % % % % % % % %

\section{Solution Methods for Differential-Algebraic Equations}

Methods for solving \acp{DAE} can be broadly categorized into two classes: (\emph{i}) direct discretization of the given system and (\emph{ii}) methods involving index reduction, which involves reformulating the system's equations so to ease the numerical solution process. While direct discretization is favored for its simplicity and straightforwardness, particularly with relatively low-index \acp{DAE} (typically less or equal than 3), index reduction approaches are essential to tackle higher-index \acp{DAE}. It is important to note that these two methods can be used alongside one the other, first by applying index reduction to come up with an index-1 or index-2 \ac{DAE} system, and secondly by applying a numerical direct discretization. As we will see, this approach keeps expressions' complexity low while exploiting the straightforwardness of direct discretization. However, there exist some classes of \acp{BVM} that allow for the direct discretization of high-index \acp{DAE} without the need for index reduction~\cite{amodio1993boundary, amodio1997parallel, amodio1998algorithm}.

\subsection{Numerical Direct Discretization}

Direct discretization is preferred due to its simplicity and computational efficiency, as index reduction methods can be time-consuming and computationally expensive, often requiring more user inputs and interventions. However, direct discretization is most effective for index-1, index-2, and index-3 Hessenberg \ac{DAE} systems. Usually, many practical \acp{DAE} fall into these categories or can be transformed into simple combinations of Hessenberg systems. Despite this, challenges may still arise, and even for these restricted classes of problems, direct application of numerical \ac{ODE} methods may lead to insufficient results and instabilities. For \acp{DAE} with an index greater than 2, employing index reduction techniques to solve the problem in a lower-index form is often considered the most effective approach. Similar considerations arise in cases resembling singularly perturbed \ac{ODE} systems, such as
%
\begin{equation}
  \begin{cases}
  \m{y}^{\prime} = \m{f}(\m{y}, \m{z}, t) \\
  \varepsilon \m{z}^{\prime} = \m{g}(\m{y}, \m{z}, t)
  \end{cases} \, \text{,}
  \label{chap1:eq:singularly_perturbed}
\end{equation}
%
where $\varepsilon$ is a small perturbation parameter. Setting $\varepsilon = 0$ reduces~\eqref{chap1:eq:singularly_perturbed} to the \acp{DAE}~\eqref{chap1:eq:semiexplicit_dae}. In general, due to the system's stiffness for small $\varepsilon$, methods designed for such \ac{ODE} system, like \ac{BDF} and Radau collocation methods, are natural choices for directly discretizing implicit \acp{DAE} of the form~\eqref{chap1:eq:dae}.

\paragraph{Backward Euler Method}

The concept of direct discretization is quite straightforward: it involves approximating $\m{x}$ and $\m{x}^{\prime}$ using a discretization technique, such as multistep methods or Runge-Kutta methods. As a first example, let us take the Backward Euler method, which is the simplest method exhibiting a stiff decay property. Applying the Backward Euler formula to $\m{x}^{\prime}$ in~\eqref{chap1:eq:dae}, we obtain a system of $n$ nonlinear equations
% TODO: controlla i pedici m n
\begin{equation}
  \m{F}\left(\m{x}_n, \dfrac{\m{x}_n - \m{x}_{n-1}}{h_n}, t_n\right) = \m{0} \quad \text{for} \quad n = 1, 2, \dots, m \, \text{.}
  \label{chap1:eq:backward_euler}
\end{equation}
%
Here, $\m{x}_n$ represents the approximation of $\m{x}(t_n)$, $h_n = t_n - t_{n-1}$ denotes the time step, $t_n$ are the time points, and $m$ is the total number of time points. Solving this nonlinear equation system recursively provides a numerical solution for~\eqref{chap1:eq:dae}.

This method is effective for index-1 \acp{DAE} and particularly suitable for stiff index-1 \acp{DAE} and stiff \acp{ODE}. However, for higher-index \acp{DAE}, including other methods, this straightforward approach may not suffice. In some cases, even seemingly simple higher-index \ac{DAE} systems with well-defined and stable solutions can pose challenges, making methods like the Backward Euler method, as well as other multistep and Runge-Kutta methods, unstable or inapplicable~\cite{ascher1998computer}. Practical difficulties may arise during the resolution of the nonlinear system~\eqref{chap1:eq:backward_euler} for $\m{x}_n$ given $\m{x}_{n-1}$, where iterative numerical methods like the Newton method are employed. Due to these implementation challenges, direct discretization of fully implicit \acp{DAE} with an index greater than 1 is generally discouraged. Despite this, for fully implicit index-1 and semi-explicit index-2 \acp{DAE}, the Backward Euler method is stable, convergent, and first-order accurate~\cite{brenan1995numerical, hairer1999stiff}.

\paragraph{Backward Differentiation Formula and Linear Multistep Methods}

Although Euler is a first-order method, achieving higher accuracy without reducing step size requires the use of higher-order methods. One such method is the constant step-size \ac{BDF}, which is applied to a general nonlinear \ac{DAE} system of the form~\eqref{chap1:eq:dae} and is given by
%
\begin{equation*}
  \m{F}\left(\m{x}_n, \dfrac{1}{\beta_0 h} \sum_{j=0}^{s} \alpha_j \m{x}_{n-j}, t_n\right) = \m{0} \quad \text{for} \quad n = 1, 2, \dots, m \, \text{,}
\end{equation*}
%
where $\beta_0$ and $\alpha_j$ for $j = 0, 1, \dots, s$ are the coefficients of the \ac{BDF} method. The $s$-step \ac{BDF} method of fixed step size $h$ has been shown to be convergent of order $\mathcal{O}(h^s)$ under certain conditions. Similar convergence results have been established for general linear multistep methods, provided their coefficients satisfy a set of order conditions, including those unique to \acp{DAE}~\cite{brenan1995numerical}. \ac{BDF} methods meet these additional requirements (refer to~\cite{brenan1995numerical} for additional information).

\paragraph{Radau Collocation and Implicit Runge-Kutta Methods}

The $s$-stage implicit Runge-Kutta method, when applied to a general nonlinear \ac{DAE} of the form~\eqref{chap1:eq:dae}, is expressed as
%
\begin{equation*}
  \begin{array}{l}
    \m{F}\left(\m{x}_{k} + h_k\displaystyle\sum_{j=1}^{s}a_{ij} \m{K}_j, \m{K}_i, t_{k} + c_i h_k\right) = \m{0} \, \text{,} \\[1em]
    \m{x}_{k+1} = \m{x}_{k} + h_k\displaystyle\sum_{i=1}^{s} b_i \m{K}_i \, \text{,}
  \end{array}
\end{equation*}
%
where $c_i$, $a_{ij}$, $b_i$ for $i = 1, 2, \dots, s$ and $j = 1, 2, \dots, s$ are the coefficients of the Runge-Kutta method, with the additional assumption that the matrix $\m{A} = (a_{ij})$ is non-singular. On the other hand, for the semi-explicit \ac{DAE}~\eqref{chap1:eq:semiexplicit_dae}, the $s$-stage implicit Runge-Kutta method reads
%
\begin{equation*}
  \begin{array}{l}
    \m{K}_i = \m{f}\left(\m{x}_{k} + h_k\displaystyle\sum_{j=1}^{s}a_{ij} \m{K}_j, \m{z}_{k}, t_{k} + c_i h_k\right) = \m{0} \, \text{,} \\[1em]
    \m{x}_{k+1} = \m{x}_{k} + h_k\displaystyle\sum_{i=1}^{s} b_i \m{K}_i \, \text{,} \\[1.75em]
    \m{0} = \m{g}\left(\m{x}_{k}, \m{z}_{k}, t_{k} + c_i h_k\right) \, \text{,}
  \end{array}
\end{equation*}
%
for $i = 1, 2, \dots, s$. It is possible to avoid the quadrature step for the algebraic variables $\m{z}$ by employing stiffly accurate methods, such as Runge-Kutta methods satisfying $b_i = a_{si}$ for $i = 1, 2, \dots, s$. These methods have additional order conditions for attaining order greater than 2, even for semi-explicit index-1 \acp{DAE}~\cite{brenan1995numerical}.

\vspace{1.0em}

Implementing direct discretization methods for \acp{DAE} faces practical challenges. These challenges encompass obtaining a consistent set of \acp{IC}, addressing the ill-conditioning of the iteration Jacobian matrix, and managing error estimation for step-size control on index-2 Hessenberg \acp{DAE}. However, for specific classes of \acp{DAE}, such as semi-explicit ones in Hessenberg form and \acp{ODE} with hidden constraints typically found in \ac{MB} mechanics, highly efficient and robust numerical methods called stabilized or projected methods exist. The core approach involves discretizing the differential equations using a suitable numerical \ac{ODE} method, together with a stabilization technique or coordinate projection step to align the numerical solution with the constraints~\cite{eichsoellner1998numerical}. For a thorough understanding of numerical aspects concerning \acp{DAE}, refer to~\cite{ascher1998computer, brenan1995numerical, hairer1999stiff}.

\subsubsection{Software for Differential-Algebraic Equations Numerical Solution}

Several software packages are available for solving \acp{DAE}, including both general-purpose and specialized tools. Some of the most widely used software packages for \ac{IVP} and \ac{BVP} \acp{DAE}'s numerical solutions are listed here below.
%
\begin{itemize}
  \setlength{\itemsep}{0.0em}
  \item \textbf{\textsc{Dassl}}, developed by Linda Petzold, utilizes \ac{BDF} formulas to solve generic index-1 \acp{DAE}. Variants tailored for large-scale problems and sensitivity analysis are also available. Specifically, \textsc{Daspk}, a later iteration of \textsc{Dassl}, can handle large Hessenberg index-2 \acp{DAE}; \textsc{Daspkadjoint}, on the other hand, implements the adjoint method for sensitivity analysis of \ac{DAE} systems~\cite{brenan1995numerical}.
  \item \textbf{\textsc{Radau5}} is a software by Ernst Hairer and Gerhard Wanner based on the 3-stage Radau collocation method~\cite{hairer1999stiff}. It tackles problems expressed as $\m{M}\m{x}^{\prime} = \m{f}(\m{x}, t)$, where $\m{M}$ is a constant, possibly singular, square matrix. \textsc{Radau5} solves \acp{DAE} up to index 3, with the identification of higher-index variables being required from the user.
  \item \textbf{\textsc{Ida}} is a submodule of the \ac{SUNDIALS} software package developed by Radu Serban and Alan Hindmarsh at \ac{LLNL}~\cite{hindmarsh2005sundials, gardner2022sundials}. \textsc{Ida} is coded in \cc{} and tackles nonlinear \acp{DAE}. It is derived from the aforementioned \Fortran{} package \textsc{Daspk}. Additionally, a related code for solving \acp{ODE} with invariants is available under the name of \textsc{Cpodes}.
  \item \textbf{\textsc{Daepack}}, is a software library created by Paul Barton and his team at \ac{MIT}~\cite{tolsma2000daepack}. While its name stands for \ac{DAE} Package, \textsc{Daepack}'s capabilities extend beyond \ac{DAE} analysis, encompassing both symbolic and numerical components for modeling and general numerical computations.
  \item \textbf{\textsc{Coldae}} was firstly presented in~\cite{bader1987new}. It employs projected Gauss collocation to solve boundary value problems for semi-explicit \acp{DAE} with an index of at most 2.
\end{itemize}

\subsection{Index Reduction Methods}
\label{chap1:sec:index_reduction_methods}

For higher-index \acp{DAE}, direct discretization methods might not offer the most efficient or effective solution. In such scenarios, index reduction methods are preferred. These methods involve reformulating the \ac{DAE} into a lower-index form, which can then be tackled using direct discretization methods or standard \ac{ODE} integration techniques, aided by projection or stabilization techniques. Specifically, the index is reduced by differentiating the algebraic constraints and substituting them into the differential equations. This process repeats until the \ac{DAE} transforms into an explicit \ac{ODE} system. As mentioned earlier, the number of iterations needed for this transformation is termed the index of the \ac{DAE}, with \acp{ODE} having an index of 0.

However, as we mentioned before, there exist multiple definitions of the index of a \ac{DAE} system, and each of these definitions is specifically tailored to a particular index reduction technique. As a consequence, there are various approaches to index reduction, whose characteristics also depend on factors like the capabilities of the available package, the user's expertise, and the specific problem at hand. Each method has its advantages and disadvantages, being more or less suitable for different types of \acp{DAE} and varying in computational costs, implementation requirements, and theoretical foundations.

\paragraph{Differential Index Reduction}

As expert readers may notice, the differential index reduction method bears close relation to symbolic computation and symbolic computation software such as \Maple{}, \Mathematica{}, and \Python{}'s library \textsc{SymPy}. These prove invaluable in executing the differentiation and substitution steps essential for reducing the differential index of a \ac{DAE} system. These tools also facilitate the derivation of the \acp{DAE}' Jacobian, crucial for numerical solution approaches. However, in some problems, symbolic computation tools can also furnish the \acp{DAE}' solution, aiding in the validation and assessment of numerical solutions obtained through numerical methods. Nevertheless, it is worth noting that symbolic computation tools are not always the optimal choice for \acp{DAE} resolution, particularly for large-scale or intricate systems. Indeed, the computational overhead of symbolic computation tools can become prohibitive in such cases, prompting a preference for other methods in solving the \ac{DAE} system.

As previously mentioned, the basic idea behind the differential index reduction method is to differentiate the algebraic constraints and substitute them into the differential equations. This process is repeated until the \ac{DAE} system is transformed into an explicit \ac{ODE} system. To do so, one can indiscriminately differentiate all equations in the \ac{DAE} system, however, this approach is not the most efficient and effective. In some cases, differentiating only a subset of the equations can lead to a more straightforward and less computationally expensive solution. This is particularly true for large-scale \acp{DAE} where differentiating all equations can lead to a significant increase in computational cost. In such cases, only algebraic equations should be differentiated. In other words, the \ac{DAE} system must be transformed into its Hessenberg form, where the algebraic variables are explicitly separated from the differential variables~\cite{shmoylova2013simplification}. Thereby, the process of isolation of algebraic variables is crucial for the successful application of the differential index reduction method.

\paragraph{Structural Index Reduction}

The most common index reduction methods are those based on the \emph{Pantelides algorithm}~\cite{pantelides1988consistent}, which falls under the umbrella of the \ac{SA} methods. However, recent advances have led to the development of more effective algorithms, such as Stephen Campbell and William Gear' method of \emph{differential arrays}~\cite{campbell1995index} and John Pryce's \emph{$\Sigma$-method}~\cite{pryce1998solving}. This latter method provides a systematic approach to index reduction and the solution of \acp{DAE} using the Taylor series. Nonetheless, it also has been proven to be a generalization of the Pantelides algorithm~\cite{pryce2001simple}.

The main concept behind \ac{SA} of \ac{DAE} consists of examining their regularity or singularity in a generic sense. This analysis is typically conducted through the representation of the system's structure using tools such as the bipartite graph, which characterizes the pattern of the system's incidence matrix. Structural regularity or singularity, as well as related properties, are then assessed based on this graph. This approach is particularly useful for sparse systems of equations, where the incidence matrices have a small proportion of non-zero entries. \ac{SA} techniques also extend to numerical equations, considering the Jacobian matrix of the system. Understanding the link between numerical regularity and structural regularity is crucial for correctly applying \ac{SA} methods. Additionally, the \ac{SA} of systems of equations can be extended to include systems with existential quantifiers~\cite{benveniste2021structural}.

In essence, the \ac{SA} of \ac{DAE} systems aims to address the following problem. Considering a \ac{DAE} system represented by equation~\eqref{chap1:eq:dae} as a set of algebraic equations with the leading variables as unknowns, the goal is to determine if this system is structurally non-singular. If it is, then given values for all the derivatives $x_k^{\prime}$ for $i = 1, 2, \dots, n$ and $k = 0, 1, \dots, d_{i-1}$, a unique leading value for the $i$ variables can be computed structurally and, in other words, the system behaves like a \ac{ODE} system. However, if the system is not structurally non-singular, additional latent equations can be derived by suitably differentiating selected equations. This process transforms the system into a \ac{ODE}-like form without altering its solution set~\cite{benveniste2021structural}.

The structural index introduced earlier serves as a metric for quantifying the number of times the system needs to be differentiated to attain a structurally non-singular form. It is important to emphasize that the structural index can be significantly higher than the differential index. In particular, \citet{reissig2000differential} highlights that the structural index of constant coefficients linear \acp{DAE}, with a differential index of 1, can be arbitrarily high, which contrasts many previous findings in the literature. This reveals that applying Pantelides' algorithm to index-1 \acp{DAE} may lead to an indefinite number of iterations and differentiations.

\paragraph{Tractability Index Reduction}

Another approach to index reduction is the so-called \emph{projector-based analysis}, which is thoroughly presented in \citet{lamour2013differential, marz2014differential}. The projector-based analysis, based on the tractability index, explores the concept that complex \acp{DAE} can consist of distinct regularity regions bordered by critical points. These regions reveal a consistent structure that can be uncovered using matrix function sequences formed with admissible projector functions and guided by constant rank conditions. Smooth solutions are expected within these regions, but crossing borders may lead to strong singularities. In essence, this method provides a systematic way of tracking the system's structure by computing an admissible matrix function sequence and monitoring rank conditions~\cite{lamour2011computational}.

More specifically, the projector-based analysis involves finding a projector that allows for the separation of the differential and algebraic components of a \ac{DAE} system with ``proper leading term''~\cite{lamour2011computational}. The projector is then used to transform the \ac{DAE} system into a lower-index form, which can be then integrated. Notice that in the case of nonlinear \acp{DAE}, the projector-based analysis is typically performed pointwise, which means that the algorithmic steps of the computation of admissible null-space projectors are performed at each time step, ensuring that the changes in the \acp{DAE} structure are accounted for at each time step. Special continuous projector-valued functions can be used to ensure that it is possible to reutilize the same projector in a wider range of time steps, thus reducing the computational cost of the method~\cite{lamour2012detecting}.

\subsection{Software for Index Reduction}

Following this brief introduction to the most widely recognized index reduction algorithms, we provide an overview of the specific index reduction algorithms found in current state-of-the-art software. The following list is limited to the most prominent solutions dedicated to dynamic system modeling and simulation that are widely adopted in both academia and industry.
%
\begin{itemize}
  \setlength{\itemsep}{0.0em}
  \item \textbf{\Matlab{}} uses the Pantelides algorithm~\cite{pantelides1988consistent} to reduce the \acp{DAE} to index-1. Alternatively, or if the latter fails, the more reliable but slower Gaussian elimination algorithm can be employed to obtain an index-0 \ac{DAE} system~\cite{matlab}.
  \item \textbf{\Modelica{}}~\cite{mattsson1997modelica, mattsson1998physical}, \Modelica{}-based software and \textbf{\ModelingToolkit{}}~\cite{modelingtoolkit} employ the Pantelides algorithm~\cite{pantelides1988consistent} along with the dummy derivatives method~\cite{mattsson1993index} to automatically perform the reduction to index-1 \acp{DAE}.
  \item \textbf{\Mathematica{}} offers a comprehensive suite of index reduction algorithms~\cite{mathematica}. It can reduce the index of \acp{DAE} using the Pantelides~\cite{pantelides1988consistent} or structural matrix~\cite{unger1995structural, chowdhry2004symbolic} methods. Additionally, it implements dummy derivatives~\cite{mattsson1993index} and projection methods for taking hidden constraints into account during numerical integration.
  \item \textbf{\Maple{}} performs symbolic index reduction within the \texttt{dsolve} function~\cite{maple}. However, the implemented algorithms are not documented or referenced. They are likely to be based on the projection method outlined in~\cite{shmoylova2013simplification}. Notably, the patents by the same authors~\cite{postma2012exact, shmoylova2012method, postma2015exact} discuss techniques for eliminating isolated parameters, extracting parameter sub-expressions from \acp{DAE}, and establishing minimal disconnected clusters of parameter sub-expressions. We would like to point out that this information is to be taken with caution, as \Maple{} does not provide specific references on this topic.
  \item \textbf{\textsc{Daesa}}, developed by Guangning Tan and Ned Nedialkov in collaboration with John Pryce, is a \Matlab{} toolbox designed for conducting \ac{SA} of \acp{DAE}. \textsc{Daesa} can analyze fully nonlinear systems, irrespective of their order or index, making it capable of determining crucial attributes such as the structural index, \acp{DOF}, constraints, and variables requiring initialization, while also proposing a suitable solution strategy~\cite{nedialkov2015algorithm, tan2016symbolic}. Additionally, it can generate a block-triangular form of the \acp{DAE}, enabling efficient block-wise solution strategies.
  \item \textbf{\textsc{Daets}}, short for \acp{DAE} by Taylor Series, is a \cpp{} package developed by Guangning Tan and Ned Nedialkov in collaboration with John Pryce~\cite{nedialkov2007solvingI, nedialkov2007solvingII, nedialkov2008solvingIII}. It is designed for tackling \acp{IVP} associated with \ac{DAE} systems. \textsc{Daets} leverages Pryce's method for the \ac{SA} of \acp{DAE}, providing a powerful means to determine the system's index, \acp{DOF}, and to precisely identify which components necessitate initial values.
  \item \textbf{\textsc{Initdae}} is a \Python{} prototype designed to calculate consistent initial values for \acp{DAE}, determining their index, as well as a condition number that aids in identifying singularities. The initialization algorithm utilizes a constrained optimization approach based on projectors, with \ac{AD}. Local structural characteristics of the \ac{DAE} system are examined through the \ac{SVD}~\cite{estvezschwarz2021initdae}
\end{itemize}
%
All the showcased software solutions offer integrators for index-0 and index-1 \acp{DAE}. Depending on the system's stiffness, users can select the most appropriate algorithm for numerically integrating the reduced-index system.

% % % % % % % % % % % % % % % % % % % % % % % % % % % % % % % % % % % % % % % %

\section{Research Objectives and Contributions}

Notably, in findings presented by the group of Roswitha M{\"a}rz, which are summarized in~\cite{lamour2013differential}, there is little mention of symbolic computation tools in the derivation of projector functions and matrix function sequences. Instead, the authors rely on numerical factorization methods with a coupled automatic differentiation method to compute the projector functions and the matrix function sequences~\cite{lamour2011computational, schwarz2015diagnosis}. However, it must be pointed out that in \citet{lamour2013differential}, authors show that the latest state-of-the-art \ac{CAS} allow the derivation of smooth symbolic projector functions, especially for simple or small-scale \acp{DAE}. This can significantly reduce the computational cost of the projector-based analysis method as it eliminates the need for numerical factorization methods and automatic differentiation. Nevertheless, the most relevant consideration arising from what has been discussed in Section~\ref{chap1:sec:index_reduction_methods} is that a neat separation between the differential and algebraic components of the \ac{DAE} system is crucial for keeping the computational cost of the index reduction method at the lowest possible level.

Starting from these observations, the primary objective of this thesis is to develop an index reduction algorithm for \acp{DAE} based on symbolic matrix factorization techniques. It must be pointed out that the algorithm is neither directly based on the projector-based analysis nor in index concepts other than the differential index. For this reason, the algorithm is limited to generic well-determined \acp{DAE} of the form~\eqref{chap1:eq:dae}, linear in the states' derivatives. The specific research objectives and contributions of this thesis to the current state-of-the-art are outlined below.
%
\begin{itemize}
  \item The development of a symbolic linear algebra package for symbolically solving linear systems of equations using matrix factorization techniques. The package, which also includes hierarchical representation techniques for expression swell mitigation and large expression management, enables the \Maple{} kernel to effectively handle larger-scale linear systems of equations. This seemingly simple contribution is crucial for the subsequent development of the symbolic index reduction algorithm for \acp{DAE} as it boosts the computational efficiency of the involved symbolic operations.
  \item The development of a novel symbolic index reduction algorithm for \acp{DAE} based on the symbolic matrix factorization techniques introduced in the previous objective. Specifically, the symbolic linear algebra package is used to separate the differential and algebraic equations of the \ac{DAE} system. A subsequent differentiation is applied to the algebraic equations to reduce the index of the \ac{DAE} system.
  \item The exploitation of the expressions' hierarchical representation structure, which is employed to mitigate the expression swelling, as a set of index-1 variables in a newly developed numerical scheme for the integration of reduced-index \acp{DAE}.
  \item Lasty, the validation of the symbolic index reduction algorithm through a set of benchmark \acp{DAE} from the literature arising in various fields, as well as the comparison of the algorithm's performance with the \Maple{}'s built-in pure symbolic-based index reduction algorithm.
\end{itemize}
%
Nonetheless, the symbolic algorithm is implemented as an open-source \Maple{} package, on the other hand, the numerical scheme is implemented as an open-source \Matlab{} toolbox. Both software are collected in the \Indigo{} toolbox~\cite{indigo}, whose dependencies are the symbolic linear algebra package \LAST{}~\cite{last} and the large expression management package \LEM{}~\cite{lem}. All the software developed in this thesis is distributed under the \ac{BSD} 3-Clause License, which allows for both academic and commercial use. The proposed symbolic algorithm and its dedicated numerical scheme are validated through a set of benchmark \acp{DAE} from the literature arising in various fields, including problems from \ac{MBD}, electrical circuit simulation, and \ac{TPPC}.

% % % % % % % % % % % % % % % % % % % % % % % % % % % % % % % % % % % % % % % %

\section{Thesis Outline}

After this introductory chapter, the remainder of this thesis is organized as follows. Chapter~\ref{chap2:symbolic_computation} provides a brief overview of the current state-of-the-art symbolic computation techniques, applications, and tools for solving foundational problems in mathematics, physics, and engineering. Then, it introduces the expression swell phenomenon and presents a symbolic computation technique used to mitigate it. The last part of the chapter is dedicated to the development of a computation package for solving large-scale linear systems of equations through symbolic linear algebra techniques, which will be used in the subsequent chapter to tackle \ac{DAE} systems. Chapter~\ref{chap3:daes} is mainly dedicated to the development of a symbolic index reduction algorithm for \acp{DAE} based on the previously introduced symbolic matrix factorization techniques. It presents the symbolic index reduction algorithm implementation, which is written in the \Maple{} language. Lastly, the numerical scheme utilized to integrate the reduced-index \acp{DAE} is validated through an index-3 \acp{DAE} example. Chapter~\ref{chap4:applications} demonstrates the application of the symbolic index reduction algorithm to a set of benchmark \acp{DAE} from the literature arising in various fields, including \ac{MBD}, electrical circuit simulation, and \ac{TPPC}. The chapter concludes with a summary of the presented algorithm performance. Finally, Chapter~\ref{chap5:conclusions} summarizes the main contributions of this thesis and discusses possible future developments.

Appendices are included at the end of the thesis to provide additional information on side topics that are not essential to the main topic of this thesis but are relevant for one of the benchmark \acp{DAE} presented in Chapter~\ref{chap4:applications}. The appendices include the presentation of a \cpp{} computational geometry library (Appendix~\ref{app1:acme}), as well as the derivation of models for tire-ground enveloping (Appendix~\ref{app2:enve}), tire contact forces (Appendix~\ref{app3:tirex}). The last appendix (Appendix~\ref{app4:trussme}) provides an overview of the development of a symbolic package for solving structures using the \ac{DSM}. Notably, these appendices correspond to the author's previous work or side projects carried out during the doctoral program.

% % % % % % % % % % % % % % % % % % % % % % % % % % % % % % % % % % % % % % % %
