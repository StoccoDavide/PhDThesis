%!TEX root = ../main.tex

\chapter{Symbolic Computation and Applications}
\label{chap2:symbolic_computation}

Il calcolo simbolico è particolarmente utile per risolvere problemi che coinvolgono manipolazioni simboliche di espressioni matematiche piuttosto che semplici valori numerici. Ecco alcuni esempi di problemi in cui il calcolo simbolico si rivela prezioso:

Derivazione e Integrazione Simbolica:
Esempio 1: Calcolare la derivata di una funzione complessa, come ad esempio .
Esempio 2: Eseguire l'integrazione simbolica di una funzione come
Algebra Simbolica:
Esempio 3: Semplificare espressioni algebriche comple
Esempio 4: Risolvere sistemi di equazioni lineari o non lineari in forma simbolica.
Equazioni Differenziali Ordinarie (EDO):
Esempio 5: Risolvere un'equazione differenziale come
Esempio 6: Trovare una soluzione generale per un sistema di EDO complesso.
Matrici e Algebra Lineare:
Esempio 7: Calcolare l'inversa di una matrice simbolica.
Esempio 8: Trovare gli autovalori e gli autovettori di una matrice.
Teoria dei Numeri:
Esempio 9: Effettuare manipolazioni simboliche con espressioni che coinvolgono numeri irrazionali o complessi.
Esempio 10: Sviluppare algoritmi simbolici per la fattorizzazione di numeri o espressioni.


Chiarisco che l'analisi numerica solitamente coinvolge metodi computazionali basati su calcoli approssimati piuttosto che manipolazioni simboliche. Tuttavia, ci sono situazioni in cui il calcolo simbolico può essere incorporato nell'analisi numerica per migliorarne l'efficienza o risolvere specifici sotto-problemi. Ecco alcuni esempi:

Analisi di Errore:
Utilizzare il calcolo simbolico per derivare formule esatte per l'errore di approssimazione in un determinato metodo numerico, ad esempio, nell'approssimazione di una derivata o di un'integrale.
Ottimizzazione di Algoritmi Numerici:
Incorporare il calcolo simbolico per semplificare e ottimizzare passaggi critici in algoritmi numerici, ad esempio, nell'ottimizzazione di formule iterative o nella semplificazione di espressioni complesse all'interno di algoritmi.
Generazione di Funzioni Approssimanti:
Utilizzare il calcolo simbolico per derivare esattamente la forma chiusa di una funzione approssimante, ad esempio, attraverso metodi di interpolazione, semplificando così i calcoli numerici successivi.
Stabilità Numerica:
Analizzare la stabilità di un algoritmo numerico utilizzando il calcolo simbolico per esaminare il comportamento asintotico delle espressioni coinvolte nel processo numerico.
Risoluzione di Problemi Lineari o Non-Lineari:
Utilizzare tecniche di calcolo simbolico per semplificare le espressioni prima di applicare metodi numerici, riducendo così il numero di operazioni computazionali e migliorando l'efficienza computazionale.
Rappresentazione Esatta di Numeri Irrazionali:
Utilizzare il calcolo simbolico per rappresentare in modo esatto numeri irrazionali coinvolti in un problema numerico, evitando così errori di approssimazione.
Analisi di Sensibilità:
Utilizzare il calcolo simbolico per derivare espressioni analitiche per la sensibilità di una soluzione numerica rispetto alle variazioni nei parametri di input.
In queste situazioni, il calcolo simbolico può essere un complemento utile all'analisi numerica, contribuendo a una migliore comprensione del problema e ottimizzando l'implementazione degli algoritmi computazionali.